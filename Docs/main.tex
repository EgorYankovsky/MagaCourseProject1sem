\documentclass[14pt,a4paper]{extreport}

\usepackage{style/style}
\usepackage{physics}
\usepackage{fancyhdr}
\usepackage{pdfpages}

\fancypagestyle{plain}{%
\fancyhf{} % clear all header and footer fields
\fancyfoot[C]{\small\thepage}}
\renewcommand{\headrulewidth}{0pt}
\renewcommand{\footrulewidth}{0pt}
\pagestyle{plain}

\makeatletter
  \def\my@tag@font{\small}
  \def\maketag@@@#1{\hbox{\m@th\normalfont\my@tag@font#1}}
  \let\amsmath@eqref\eqref
  \renewcommand\eqref[1]{{\let\my@tag@font\relax\amsmath@eqref{#1}}}
\makeatother

\usepackage{titletoc}
\titlecontents{chapter}[0em]{\bfseries}{\thecontentslabel.\hspace{1em}}{}{\titlerule*[1pc]{.}\contentspage}
\titlecontents{section}[1.25em]{}{\thecontentslabel.\hspace{1em}}{}{\titlerule*[1pc]{.}\contentspage}
\titlecontents{subsection}[2.5em]{}{\thecontentslabel.\hspace{1em}}{}{\titlerule*[1pc]{.}\contentspage}

\begin{document}

\includepdf[pages=-]{Title.pdf}
\includepdf[pages=-]{Task_list.pdf}

% Отключение нумерации страниц
\pagenumbering{gobble}

\chapter*{Аннотация}

Отчёт 91 с., 4 ч., 24 рис., 26 табл., 13 источников, 1 прил.

ЧИСЛЕННОЕ МОДЕЛИРОВАНИЕ ЭЛЕКТРОМАГНИТНОГО \\ ПОЛЯ, МЕТОД КОНЕЧНЫХ ЭЛЕМЕНТОВ, МНОГОЭТАПНАЯ СХЕМА РАЗДЕЛЕНИЯ ПОЛЕЙ

Цель работы: разработка программы для численного моделирования нестационарного электромагнитного поля в трёхмерной среде, создаваемого индукционным источником тока, при помощи многоэтапной схемы разделения полей.

В процессе работы был разработан и протестирован  программный модуль численного моделирования электромагнитного поля с помощью многоэтапной схемы разделения полей.

С помощью программы проводилось исследование поведения поля в приповерхностных слоях земной коры с различными значениями удельной электропроводности горизонтально-слоистой среды и аномальных объектов. 

\newpage

\tableofcontents
\newpage

% Включение нумерации страниц
\pagenumbering{arabic}
\setcounter{page}{6}
\chapter*{Введение}

\addcontentsline{toc}{chapter}{Введение}

С увеличением потребности в природных ресурсах с конца $\text{XIX}$-ого -- начала $\text{XX}$-ого века развивались методы поиска и исследования земных пород и руд. Одним из наиболее распространенных методов является электроразведка. Конкретно в электроразведке сейчас насчитывается свыше пятидесяти различных методов и модификаций, предназначенных как для глубинных исследований, так и для изучения верхней части разреза земной коры. Одним из наиболее распространённым в наши дня является индукционный метод. Его принцип заключается в следующем: под влиянием переменного электрического или магнитного поля в земле за счет феномена магнитной индукции возникает электромагнитное поле. Зная точно параметры источника поля, можно измерять различные электрические и магнитные компоненты индуцированного поля, восстанавливая по ним значения параметров среды.

Помимо возможности нахождения параметров среды, что является обратной задачей по определению, можно изучать и поведение самого индукционного поля. Зная значения силы тока на источнике и физические характеристики верхних слоёв земной коры можно смоделировать электромагнитное поле и изучать характер его поведения в зависимости, например, от количества неоднородностей в земной среде или параметров горизонтально-слоистой среды Земли \cite{1}.

Однако на тот момент аналитических методов для моделирования электромагнитных полей в трёхмерном пространстве не существовало. Достаточно быстро решать такие задачи стало возможно лишь с развитием ЭВМ -- в 30-х -- 70-х годах $\text{XX}$--го века. Использование компьютеров позволило численно моделировать сложные поля, для которых было необходимо преобразовывать их таким образом, чтобы разделить аномалии в зависимости от глубины расположения источников поля и обособлять такие изменения поля, которые соответствуют аномалиям тел простейших форм \cite{2}. 

В это же время для решения сложных уравнений в частных производных, описывающих большинство всех физический процессов в природе, были разработаны такие методы как: метод конечных элементов (МКЭ), метод конечных объемов (МКО) и метод конечных разностей (МКР). МКР известен своей простотой в реализации и используется при решении очень большого класса задач математической физики. Однако данный метод имеет ряд существенных недостатков, например, невозможность решения задачи на объектах, имеющих сложную геометрию. Куда более универсальными численными методами решения задач математической физики являются МКО и МКЭ. Оба метода применяются для решения задач аэрогидродинамики, электростатики, упругости и прочности материалов. Данные методы позволяют решать задачи для предметов любой геометрической сложности, а также находить значение искомой функции в любом месте расчетной области без дополнительного применения интерполяции полученного результата. Сегодня есть очень много профессионального коммерческого программного обеспечения для анализа проблем методом конечных элементов или контрольных объёмов, например американская ANSYS или российская Штуцер-МКЭ.

При численном моделировании электромагнитных полей, большее предпочтение отдаётся МКЭ нежели МКО, поскольку при решении задачи методом конечных элементов можно воспользоваться модификацией данного метода - векторным МКЭ. Данная модификация позволяет куда лучше найти решение задачи при наличии, например, разрывности напряженности электрического поля и удельной электрической проводимости среды.

В данной работе будет рассмотренна возможность расчёта нестационарного электромагнитного поля, создаваемого индукционным источником при использовании как скалярного, так и векторного МКЭ, а также с применением многоэтапной технологии разделения полей.

Исследования будем проводить на области размером $\Omega \in [-55000; 55000]_x \times [-55000; 55000]_y \times [-25000; 25000]_z$. Данная область включает в себя несколько разных слоев земных пород, имеющих различные удельные значения физических величин. Также рассмотрим процесс добавления нескольких аномальных пород, характерных различным земным рудам.

При написании программы для выпускной квалификационной работы использовались следующие языки программирования: для математических расчётов -- C$\#$ 12 на платформе .NET 8.0, для визуализации полученных результатов -- Python 3.12.2 с пакетами matplotlib версии 3.8.2 и numpy версии 1.26.4. 

\chapter{Постановка задачи}

\section{Аппарат математического моделирования}

Математическая модель, описывающая поведение электромагнитного поля в пространстве, известна в наши дни, как система уравнений Максвелла. Она позволяет описывать взаимосвязь сразу нескольких физических величин: напряжённости электрического $\overrightarrow{\textbf{E}}$ и магнитного $\overrightarrow{\textbf{H}}$ полей, а также индукцию магнитного поля $\overrightarrow{\textbf{B}}$. Большинство вычислительных задач электромагнетизма базируются на дифференциальной форме системы уравнений Максвелла:

\begin{equation} \label{eq_1_1}
	\text{rot} \overrightarrow{\textbf{H}} = \overrightarrow{\textbf{J}^{\text{ст}}} + \sigma \overrightarrow{\textbf{E}} + \frac{\partial \left(\varepsilon \overrightarrow{\textbf{E}} \right)}{\partial t},
\end{equation}

\begin{equation} \label{eq_1_2}
	\text{rot} \overrightarrow{\textbf{E}} = - \frac{\partial \overrightarrow{\textbf{B}}}{\partial t},
\end{equation}

\begin{equation} \label{eq_1_3}
	\text{div} \overrightarrow{\textbf{B}} = 0,
\end{equation}

\begin{equation} \label{eq_1_4}
	\text{div} \varepsilon \overrightarrow{\textbf{E}} = \rho,
\end{equation}
где $\overrightarrow{\textbf{J}^{\text{ст}}}$ -- вектор плотностей сторонних токов, $\sigma$ -- удельная электрическая проводимость среды, $\varepsilon$ -- диэлектрическая проницаемость среды, а $\rho$ -- объёмная плотность стороннего электрического заряда.

Основное преимущество использования системы уравнений (\ref{eq_1_1}) -- (\ref{eq_1_4}) в дифференциальной форме, заключается в возможности учитывать нелинейность, анизотропию и другие нетривиальные аспекты среды \cite{3}. 

Пусть электромагнитное поле возбуждается индукционным источником. В таком случае, при отсутствии аномальных объектов, будем решать задачу в цилиндрических координатах. Источник поля в таком случае описывается точкой, расположенной на некотором расстоянии, достаточно далёком от границы расчётной области. Тогда при условии однородности среды по магнитной проницаемости и отстутствия токов смещения электромагнитное поле полностью описывается одной компонентой $A_{\varphi} = A_{\varphi}(r, z, t)$ вектор-потенциала $\overrightarrow{\textbf{A}}$. Функция $A_{\varphi}(r, z, t)$ может быть найдена из решения двумерного уравнения (\ref{eq_1_5}):

\begin{equation} \label{eq_1_5}
	-\frac{1}{\mu_0} \Delta A_{\varphi} + \frac{A_{\varphi}}{\mu_0 r^2} + \sigma \frac{\partial A_{\varphi}}{\partial t} = J_{\varphi},
\end{equation}
где $\mu_0 = 4 \cdot \pi \cdot 10^{-7} = 1.25663753 \cdot 10^{-6}$ Гн/м -- магнитная постоянная, $J_{\varphi}$ - источник стороннего тока, описываемый дельта-функцией, равной 1 в одной из подобласти, описывающей источник поля, и 0 во всех остальных \cite{4}. Удельную электропроводность $\sigma$ представим в виде кусочно-постоянной функции, описывающей физические характеристики горизонтально-слоистой среды. Потребуем, чтобы на всех границах было главное краевое условие $\left.A_{\varphi}(r, z, t)\right|_s = 0$. Тогда решение задачи (\ref{eq_1_5}) с главными однородными условиями на границах будем называть первичным или нормальным полем.

Решением задачи на оценку влияния аномальных объектов в горизонтально-слоистой среде будем называть вторичным (добавочным) полем. Также, как и в (\ref{eq_1_5}) потребуем на всех границах главное однородное краевое условие $\overrightarrow{\textbf{A}} \times \overrightarrow{\textbf{n}} |_s = 0$. Тогда, нестационарный процесс, возникающий после выключения источника тока в круглой обмотке, описывается следствием из уравнения (\ref{eq_1_1}) \cite{5}:

\begin{equation} \label{eq_1_6}
	\text{rot} \left( \frac{1}{\mu_o} \text{rot} \overrightarrow{\textbf{A}}^{+} \right) + \sigma \frac{\partial \overrightarrow{\textbf{A}}^{+}}{\partial t} = (\sigma - \sigma_n) \overrightarrow{\textbf{E}}^n,
\end{equation}
где $\sigma_n$ -- значение удельной электрической проводимости среды на нормальном слое, $\overrightarrow{\textbf{E}}^n$ -- напряжённость первичного электрического поля, $\overrightarrow{\textbf{A}}^{+}$ -- значение вектор-потенциала на добавочном поле.

\section{Описание расчётной области}

Пусть у нас имеется расчётная область, геометрически представленная в виде параллелепипеда: $\Omega \in [-55000, 55000]_x \times [-55000, 55000]_y \times [-25000, 25000]_z$. Внутри неё имеются слои воздуха, и некоторых пород верхних слоёв земной коры. Тогда половина продольного диагонального среза горизонтально-слоистой среды изображена на рисунке \ref{fig:example}. Будем её использовать в качестве расчётной области для двумерной задачи. В среде, обозначенной коричневым цветом задано значение $\sigma_1 = 0.01$ См/м, в бледной $\sigma_2 = 0.005$ См/м и в зелёной $\sigma_3 = 0.001$ См/м. Поскольку воздух является диэлектриком, значение удельной электропроводности для него $\sigma_{\text{возд.}} = 0$ См/м.

\begin{figure}
	\centering
	\vspace*{0.7cm}
	\includegraphics[width=1.0\linewidth]{images/"Figure_example".png}
	\caption{Срез горизонтально-слоистой среды}
	\label{fig:example}
\end{figure}
\chapter{Теоретическая часть}

\section{Многоэтапная схема разделения поля}


Предлагаемый в работе подход к численному моделированию основан на технологии разделения полей, позволяющей существенно сократить вычислительные затраты. В рассматриваемой задаче под неоднородностями (аномалиями) будем понимать трёхмерные геологические объекты, существенно отличные от сопротивления вмещающей горизонтально-слоистой среды \cite{6}. Сначала решение ищется на поле, создаваемом в среде, максимально упрощённой относительно исходной. Её решение, которое может быть меньшей размерности, берётся в качестве основного поля первого уровня. На основе этого поля решается задача на добавочное поле, в которую включается часть неоднородностей исходной задачи, дающих максимальный вклад в искомое решение. Используемая для нахождения этого добавочного поля сетка строится так, чтобы максимально учесть влияние источников, порождённых включёнными в на этом этапе неоднородностями \cite{7}.

Далее в качестве основного поля будет учитываться сумма основного и добавочного на предыдущем этапе выделения. Новое добавочное поле будет формироваться из учёта следующих по влиянию на решение исходной задачи. Процесс можно продолжать до тех пор, пока не будут учтены все неоднородности среды.

\section{Вариационная постановка двумерной задачи}

Перед тем, как приступить к решению задачи на нормальное поле, необходимо перевести уравнение (\ref{eq_1_5}) в вариационную форму. В основе использования МКЭ лежит вариационная постановка, в которой решение краевой задачи заменяется минимизацией функционала невязки. Областью определения этого функционала обычно является Гильбертово пространство функций $H^m$, содержащее в качестве одного из своих элементов решение данной краевой задачи. Потребуем, чтобы невязка $R(A_{\varphi}) = -\frac{1}{\mu_0} \Delta A_{\varphi} + \frac{A_{\varphi}}{\mu_0 r^2} + \sigma \frac{\partial A_{\varphi}}{\partial t} - J_{\varphi}$ дифференциального уравнения (\ref{eq_1_5}) была ортогональна в смысле скалярного произведения пространства $L^2(\Omega) \equiv H_0$ некоторому пространству $\Phi$ функций $v$, которое называется пространством пробных функций, т.е.:

\begin{equation} \label{eq_2_1}
	\int \limits_{\Omega} \left( -\frac{1}{\mu_0} \Delta A_{\varphi} + \frac{A_{\varphi}}{\mu_0 r^2} + \sigma \frac{\partial A_{\varphi}}{\partial t} - J_{\varphi}\right) v d\Omega = 0,
\end{equation}

\begin{equation} \label{eq_2_2}
	\int \limits_{\Omega} \left( -\frac{1}{\mu_0} \Delta A_{\varphi} + \frac{A_{\varphi}}{\mu_0 r^2} + \sigma \frac{\partial A_{\varphi}}{\partial t} \right) v d\Omega = \int \limits_{\Omega} J_{\varphi} v d\Omega.
\end{equation}

Используя формулу Грина, получим:

\begin{equation} \label{eq_2_3}
\begin{gathered}
	\int \limits_{\Omega}  \frac{1}{\mu_0} \text{grad} A_{\varphi} \cdot \text{grad} v d\Omega - \int \limits_{S} \frac{1}{\mu_0} \frac{\partial A_{\varphi}}{\partial n} v dS + \int \limits_{\Omega} \frac{A_{\varphi}}{\mu_0 r^2} d \Omega + \\ + \int \limits_{\Omega} \sigma \frac{\partial A_{\varphi}}{\partial t}  v d\Omega = \int \limits_{\Omega} J_{\varphi} v d\Omega.
\end{gathered}
\end{equation}

Так как пространство пробных функций $H_0^1$ имеет след 0, то слагаемое $\int \limits_{S} \frac{1}{\mu_0} \frac{\partial A_{\varphi}}{\partial n} v dS$ не оказывает никакого вклада в (\ref{eq_2_3}). Таким образом для (\ref{eq_1_5}) получим уравнение в слабой форме:

\begin{equation} \label{eq_2_4}
	\int \limits_{\Omega}  \frac{1}{\mu_0} \text{grad} A_{\varphi} \cdot \text{grad} v d\Omega + \int \limits_{\Omega} \frac{A_{\varphi}}{\mu_0 r^2} v d \Omega + \int \limits_{\Omega} \sigma \frac{\partial A_{\varphi}}{\partial t}  v d\Omega = \int \limits_{\Omega} J_{\varphi} v d\Omega.
\end{equation}

\section{Конечноэлементная дискретизация двумерной задачи}

Сетку для решения задачи будем строить используя прямоугольные элементы. При использовании многоэтапной схемы разделения поля для первичного слоя допускается, что сетка может быть равномерной, не сгущающейся к неоднородностям среды. Поэтому рассмотрим сетку на 26878 узлов, имеющей 302 узла по оси $r$ и 89 узлов по оси $z$.

Для решения задачи будем использовать билинейные базисные функции, задаваемые линейными функциями на $\Omega_{ps} = [r_p, r_{p + 1}] \times [z_s, z_{s + 1}]$ следующего вида: 


\begin{equation} \label{eq_2_5}
	\begin{cases}
		& \hat{\psi_1}(r,z) = R_1(r)Z_1(z), \\
		& \hat{\psi_2}(r,z) = R_2(r)Z_1(z), \\
		& \hat{\psi_3}(r,z) = R_1(r)Z_2(z), \\
		& \hat{\psi_4}(r,z) = R_2(r)Z_2(z). \\
	\end{cases}
\end{equation}
где:

\begin{equation} \label{eq_2_6}
	\begin{cases}
		& R_1(r) = \frac{r_{p + 1} - r}{r_{p + 1} - r_p}, \\
		& R_2(r) = \frac{r - r_p}{r_{p + 1} - r_p}, \\
		& Z_1(z) = \frac{z_{p + 1} - z}{z_{p + 1} - z_p}, \\
		& Z_2(z) = \frac{z - z_p}{z_{p + 1} - z_p}. \\
	\end{cases}
\end{equation}


\section{Построение матриц масс и жёсткости для двумерной задачи}

Будем считать, что функции $A_{\varphi}$ и $v$ в вариационном уравнении (\ref{eq_2_4}) являются компонентами конечноэлементного функционального пространства, натянутого на базисные функции $\hat{\psi_j}$, где $j = \overline{1, n}$, т.е.

\begin{equation} \label{eq_2_7}
	\begin{cases}
		& A_{\varphi} = \displaystyle\sum_{j=1}^{n} q_j^{A_{\varphi}} \hat{\psi_j}, \\
		& v = \displaystyle\sum_{j=1}^{n} q_j^v \hat{\psi_j},
	\end{cases}
\end{equation}
где $q_j^{A_{\varphi}}$ -- веса в разложении функции $u$ по базисным функциями $\hat{\psi_j}$, а $q_j^v$ -- веса в разложении функции $v$ по тем же базисным функциями $\hat{\psi_j}$. Нетрудно убедиться, что с учётом разложения (\ref{eq_2_7}) вариационное уравнение (\ref{eq_2_4}) эквивалентно системе уравнений

\begin{equation} \label{eq_2_8}
\begin{gathered}
	\displaystyle\sum_{j=1}^{n} \left( \int \limits_{\Omega}  -\frac{1}{\mu_0} \text{grad} \hat{\psi_i} \cdot \text{grad} \hat{\psi_j} d\Omega + \int \limits_{\Omega} \frac{\hat{\psi_i} \hat{\psi_j}}{\mu_0 r^2} d \Omega \right) + \\ + \displaystyle\sum_{j=1}^{n} \int \limits_{\Omega} \sigma \frac{\partial \hat{\psi_i}}{\partial t}  \hat{\psi_j} d\Omega \cdot q_j^u = \int \limits_{\Omega} J_{\varphi} \hat{\psi_i} d\Omega.
\end{gathered}
\end{equation}

Так как задача решается в цилиндрических координатах, то в уравнении (\ref{eq_2_8}) $d \Omega = r dr dz$, где $r$ - якобиан перехода от декартовых к цилиндрическим координатам.

\begin{equation} \label{eq_2_9}
\begin{gathered}
	\displaystyle\sum_{j=1}^{n} \left( \int \limits_{r} \int \limits_{z} r \frac{1}{\mu_0} \text{grad} \hat{\psi_i} \cdot \text{grad} \hat{\psi_j} dr dz + \int \limits_{r} \int \limits_{z} r  \frac{\hat{\psi_i}}{\mu_0 r^2} \hat{\psi_j} dr dz \right) \cdot q_j^u + \\ + \displaystyle\sum_{j=1}^{n} \int \limits_{r} \int \limits_{z} r  \sigma \frac{\partial \hat{\psi_i}}{\partial t}  \hat{\psi_j} dr dz \cdot q_j^u = \int \limits_{r} \int \limits_{z} r  J_{\varphi} \hat{\psi_i} dr dz.
\end{gathered}
\end{equation}

Рассмотрим аппроксимацию по пространству $(r, z)$ в уравнении (\ref{eq_2_9}) для произвольных $i$ и $j$

\begin{equation} \label{eq_2_10}
\begin{gathered}
	\left( \int \limits_{r} \int \limits_{z} r \frac{1}{\mu_0} \text{grad} \hat{\psi_i} \cdot \text{grad} \hat{\psi_j} dr dz + \int \limits_{r} \int \limits_{z} r  \frac{\hat{\psi_i}}{\mu_0 r^2} \hat{\psi_j} dr dz \right) \cdot q_j^u = \\ = \int \limits_{r} \int \limits_{z} r  J_{\varphi} \hat{\psi_i} dr dz.
\end{gathered}
\end{equation}

Пусть $\hat{\psi_i} = R_i(r) \cdot Z_i(z)$, а $\hat{\psi_j} = R_j(r) \cdot Z_j(z)$, тогда

\begin{equation} \label{eq_2_11}
\begin{gathered}
	\left( \int \limits_{r} \int \limits_{z} r \frac{1}{\mu_0} \text{grad} (R_i Z_i) \cdot \text{grad} (R_j Z_j) dr dz + \int \limits_{r} \int \limits_{z} r  \frac{R_i Z_i}{\mu_0 r^2} R_j Z_j dr dz \right) \cdot q_j^u = \\ = \int \limits_{r} \int \limits_{z} r  J_{\varphi} R_i Z_i dr dz.
\end{gathered}
\end{equation}

Преобразовав (\ref{eq_2_11}) получим следующее

\begin{equation} \label{eq_2_12}
\begin{gathered}
	\left(\frac{1}{\mu_0} \left(\int \limits_{r} r \frac{d R_i}{dr} \frac{d R_j}{dr} dr \cdot \int \limits_{z} Z_i Z_j dz + \int \limits_{r} r R_i R_j dr \cdot \int \limits_{z} \frac{d Z_i}{dz} \frac{d Z_j}{dz} dz\right) \right) q_j^u + \\ + \left( \frac{1}{\mu_0} \int \limits_{r} \frac{1}{r} R_i R_j dr \int \limits_{z} Z_i Z_j dz \right) q_j^u = \int \limits_{r} \int \limits_{z} r  J_{\varphi} R_i Z_i dr dz.
\end{gathered}
\end{equation}

Получим локальные матрицы масс $\hat{\textbf{M}}$ и жёсткости $\hat{\textbf{G}}$, а также локальный вектор правой части $\hat{\textbf{b}}$. Поскольку на одном элементе для аппроксимации билинейными базисными функциями необходимо 4 узла, то локальные матрицы будут иметь размерность 4 $\times$ 4, а векторы 4 $\times$ 1. Заметим, что каждый интеграл в уравнении (\ref{eq_2_12}) является компонентой интеграла, лежащего в основе построения локальных матриц масс и жёсткости для одномерных задач (\ref{eq_2_13}) -- (\ref{eq_2_17}).

\begin{equation} \label{eq_2_13}
	\hat{\textbf{G}}_r^{1D} = \frac{r_k + \frac{h_k}{2}}{h_k} \left(
	\begin{array}{rr}
		1 & -1\\
		-1 &  1\\
	\end{array}
	\right),
\end{equation}

\begin{equation} \label{eq_2_14}
	\hat{\textbf{M}}_r^{1D} = \frac{\hat{\gamma} h_k}{6} \left( r_k \left(
	\begin{array}{rr}
		2 & 1\\
		1 & 2\\
	\end{array}
	\right) + \frac{h_k}{2} \left(
	\begin{array}{rr}
		1 & 1\\
		1 & 3\\
	\end{array}
	\right) \right),
\end{equation}


\begin{equation} \label{eq_2_15}
\begin{gathered}
	\hat{\textbf{M}}_{rr}^{1D} = \ln\left(1 + \frac{1}{d}\right)
	\left(
	\begin{array}{cc}
		(1+d)^2 & -d(1+d)\\
		-d(1+d) &  d^2\\
	\end{array}
	\right)
	-d
	\left(
	\begin{array}{rr}
		1 & -1\\
		-1 & 1\\
	\end{array}
	\right) + \\
	+ \frac{1}{2}
	\left(
	\begin{array}{rr}
		-3 & 1\\
		1 & 1\\
	\end{array}
	\right),
\end{gathered}
\end{equation}
где $d = \frac{r_k}{h_k}$.


\begin{equation} \label{eq_2_16}
	\hat{\textbf{G}}_z^{1D} = \frac{\hat{\lambda}}{h_k} \left(
	\begin{array}{rr}
		1 & -1\\
		-1 &  1\\
	\end{array}
	\right),
\end{equation}

\begin{equation} \label{eq_2_17}
	\hat{\textbf{M}}_z^{1D} = \frac{\hat{\gamma} h_k}{6} \left(
	\begin{array}{rr}
		2 & 1\\
		1 & 2\\
	\end{array}
	\right).
\end{equation}

Тогда элементы верхнего треугольника матрицы жесткости для двумерных задач, можем представить в виде:

\begin{equation*}
	\begin{array}{ll}
		\hat{\textbf{G}}_{11} = \left(\hat{\textbf{G}}^{1D}_{r11}\hat{\textbf{M}}^{1D}_{z11} + \hat{\textbf{M}}^{1D}_{r11}\hat{\textbf{G}}^{1D}_{z11}\right), & \hat{\textbf{G}}_{12} = \left(\hat{\textbf{G}}^{1D}_{r12}\hat{\textbf{M}}^{1D}_{z11} + \hat{\textbf{M}}^{1D}_{r12}\hat{\textbf{G}}^{1D}_{z11}\right),\\
		\hat{\textbf{G}}_{13} = \left(\hat{\textbf{G}}^{1D}_{r11}\hat{\textbf{M}}^{1D}_{z12} + \hat{\textbf{M}}^{1D}_{r11}\hat{\textbf{G}}^{1D}_{z12}\right), & \hat{\textbf{G}}_{14} = \left(\hat{\textbf{G}}^{1D}_{r12}\hat{\textbf{M}}^{1D}_{z12} +\hat{\textbf{M}}^{1D}_{r12}\hat{\textbf{G}}^{1D}_{z12}\right),\\
		\hat{\textbf{G}}_{22} = \left(\hat{\textbf{G}}^{1D}_{r22}\hat{\textbf{M}}^{1D}_{z11} + \hat{\textbf{M}}^{1D}_{r22}\hat{\textbf{G}}^{1D}_{z11}\right), & \hat{\textbf{G}}_{23} = \left(\hat{\textbf{G}}^{1D}_{r21}\hat{\textbf{M}}^{1D}_{z12} +\hat{\textbf{M}}^{1D}_{r21}\hat{\textbf{G}}^{1D}_{z12}\right),\\
		\hat{\textbf{G}}_{24} = \left(\hat{\textbf{G}}^{1D}_{r22}\hat{\textbf{M}}^{1D}_{z12} + \hat{\textbf{M}}^{1D}_{r22}\hat{\textbf{G}}^{1D}_{z12}\right), & \hat{\textbf{G}}_{33} = \left(\hat{\textbf{G}}^{1D}_{r11}\hat{\textbf{M}}^{1D}_{z22} + \hat{\textbf{M}}^{1D}_{r11}\hat{\textbf{G}}^{1D}_{z22}\right),\\
		\hat{\textbf{G}}_{34} = \left(\hat{\textbf{G}}^{1D}_{r12}\hat{\textbf{M}}^{1D}_{z22} + \hat{\textbf{M}}^{1D}_{r12}\hat{\textbf{G}}^{1D}_{z22}\right), & \hat{\textbf{G}}_{44} = \left(\hat{\textbf{G}}^{1D}_{r22}\hat{\textbf{M}}^{1D}_{z22} + \hat{\textbf{M}}^{1D}_{r22}\hat{\textbf{G}}^{1D}_{z22}\right).\\
	\end{array}
\end{equation*}

Верхний треугольник элементов матрицы масс, для слагаемого с коэффициентом $\frac{1}{r^2}$ может быть представлен в виде:

\begin{equation*}
	\begin{array}{ll}
		\hat{\textbf{M}}_{11} = \hat{\textbf{M}}^{1D}_{rr11}\hat{\textbf{M}}^{1D}_{z11}, & \hat{\textbf{M}}_{12} = \hat{\textbf{M}}^{1D}_{rr12}\hat{\textbf{M}}^{1D}_{z11},\\
		\hat{\textbf{M}}_{13} = \hat{\textbf{M}}^{1D}_{rr11}\hat{\textbf{M}}^{1D}_{z12}, & \hat{\textbf{M}}_{14} = \hat{\textbf{M}}^{1D}_{rr12}\hat{\textbf{M}}^{1D}_{z12},\\
		\hat{\textbf{M}}_{22} = \hat{\textbf{M}}^{1D}_{rr22}\hat{\textbf{M}}^{1D}_{z11}, & \hat{\textbf{M}}_{23} = \hat{\textbf{M}}^{1D}_{rr21}\hat{\textbf{M}}^{1D}_{z12},\\
		\hat{\textbf{M}}_{24} = \hat{\textbf{M}}^{1D}_{rr22}\hat{\textbf{M}}^{1D}_{z12}, & \hat{\textbf{M}}_{33} = \hat{\textbf{M}}^{1D}_{rr11}\hat{\textbf{M}}^{1D}_{z22},\\
		\hat{\textbf{M}}_{34} = \hat{\textbf{M}}^{1D}_{rr12}\hat{\textbf{M}}^{1D}_{z22}, & \hat{\textbf{M}}_{44} = \hat{\textbf{M}}^{1D}_{rr22}\hat{\textbf{M}}^{1D}_{z22}.\\
	\end{array}
\end{equation*}


Верхние треугольники элементов матрицы масс, для слагаемых с коэффициентом $\sigma$ могут быть представлены в виде:

\begin{equation*}
	\begin{array}{ll}
		\hat{\textbf{C}}_{11} = \hat{\textbf{M}}^{1D}_{r11}\hat{\textbf{M}}^{1D}_{z11}, & \hat{\textbf{C}}_{12} = \hat{\textbf{M}}^{1D}_{r12}\hat{\textbf{M}}^{1D}_{z11},\\
		\hat{\textbf{C}}_{13} = \hat{\textbf{M}}^{1D}_{r11}\hat{\textbf{M}}^{1D}_{z12}, & \hat{\textbf{C}}_{14} = \hat{\textbf{M}}^{1D}_{r12}\hat{\textbf{M}}^{1D}_{z12},\\
		\hat{\textbf{C}}_{22} = \hat{\textbf{M}}^{1D}_{r22}\hat{\textbf{M}}^{1D}_{z11}, & \hat{\textbf{C}}_{23} = \hat{\textbf{M}}^{1D}_{r21}\hat{\textbf{M}}^{1D}_{z12},\\
		\hat{\textbf{C}}_{24} = \hat{\textbf{M}}^{1D}_{r22}\hat{\textbf{M}}^{1D}_{z12}, & \hat{\textbf{C}}_{33} = \hat{\textbf{M}}^{1D}_{r11}\hat{\textbf{M}}^{1D}_{z22},\\
		\hat{\textbf{C}}_{34} = \hat{\textbf{M}}^{1D}_{r12}\hat{\textbf{M}}^{1D}_{z22}, & \hat{\textbf{C}}_{44} = \hat{\textbf{M}}^{1D}_{r22}\hat{\textbf{M}}^{1D}_{z22}.\\
	\end{array}
\end{equation*}

Так как мы имеем сосредоточенный в точке источник $J_{\varphi}$, то получим следующее:

\begin{equation} \label{eq_2_18}
	\int \limits_{\Omega} J_{\phi} \delta d \Omega = 
	\left \{ \begin{aligned}
		& 1, p \in \Omega_{\epsilon}\\
		& 0,    p \notin \Omega_{\epsilon}\\
	\end{aligned} \right.
\end{equation}

Преобразуем теперь нестационарную составляющую в уравнении (\ref{eq_2_4}) в матричную форму. Для аппроксимации задачи по времени будем использовать трёхслойную неявную схему. Тогда искомое решение $A_{\varphi}$ на интервале $(t_{j-2}, t_j)$ представим в следующем виде:

\begin{equation} \label{eq_2_19}
	A_{\varphi}(r, z, t) \approx A_{\varphi}^{j - 2}(r, z) \eta_2(t)^j + A_{\varphi}^{j - 1}(r, z) \eta_1(t)^j + A_{\varphi}^{j}(r, z) \eta_0(t)^j,
\end{equation}
где $\eta_2(t)^j$, $\eta_1(t)^j$, $\eta_0(t)^j$ -- базисные квадратичные полиномы Лагранжа, которые записываются в виде (\ref{eq_2_20}).

\begin{equation} \label{eq_2_20}
	\begin{cases}
		& \eta_2(t)^j = \frac{(t-t_{j-1})(t - t_j)}{(t_{j-1} - t_{j-2}) (t_j - t_{j-2})}, \\
		& \eta_1(t)^j = -\frac{(t-t_{j-2})(t - t_j)}{(t_{j-1} - t_{j-2}) (t_j - t_{j-1})}, \\
		& \eta_0(t)^j = \frac{(t-t_{j-2})(t - t_{j - 1})}{(t_{j} - t_{j-2}) (t_j - t_{j-1})}.
	\end{cases}
\end{equation}

Подставим выражение (\ref{eq_2_19}) в нестационарное слагаемое уравнения (\ref{eq_2_4}) на временном слое $t=t_j$:

\begin{equation} \label{eq_2_23}
\begin{gathered}
	\frac{\partial}{\partial t} \left(A_{\varphi}^{j - 2}(r, z) \eta_2^j(t) + A_{\varphi}^{j - 1}(r, z) \eta_1^j(t) + A_{\varphi}^{j}(r, z) \eta_0^j(t)\right) |_{t=t_j} = \\ = \tau_2 A_{\varphi}^{j - 2}(r, z) + \tau_1 A_{\varphi}^{j - 1}(r, z) +\tau_0 A_{\varphi}^{j}(r, z),
\end{gathered}
\end{equation}
где

\begin{equation} \label{eq_2_24}
	\begin{cases}
		& \tau_2 = \left.\frac{\partial \eta_2^j(t)}{\partial t}\right|_{t=t_j} = \frac{t_j - t_{j-1}}{(t_{j-1} - t_{j-2}) (t_j - t_{j-2})}, \\
		
		& \tau_1 = \left.\frac{\partial \eta_1^j(t)}{\partial t}\right|_{t=t_j} = -\frac{t_j - t_{j-2}}{(t_{j-1} - t_{j-2}) (t_j - t_{j-1})}, \\
		
		& \tau_0 = \left.\frac{\partial \eta_0^j(t)}{\partial t}\right|_{t=t_j} = \frac{(t_j - t_{j-2}) + (t_{j} - t_{j-1})}{(t_{j} - t_{j-2}) (t_j - t_{j-1})}.
	\end{cases}
\end{equation}


Выполняя конечноэлементную аппроксимацию краевой задачи для уравнения (\ref{eq_2_4}), получим СЛАУ следующего вида:

\begin{equation} \label{eq_2_27}
	\left(\tau_0 \textbf{C} + \textbf{G} + \textbf{M}\right) \textbf{q}^j = \textbf{b} - \tau_2 \textbf{C} \textbf{q}^{j-2} + \tau_1 \textbf{C} \textbf{q}^{j-1}.
\end{equation}

\section{Связь компонент электромагнитного поля в декартовой и цилиндрической системе координат}

Как мы уже выяснили, решение задачи в $(r, z)$ координатах можно использовать в качестве нормального поля. Перед тем, как перейти к трёхмерной постановке задачи, найдем значение $E^0_{\varphi}$, через следующее преобразование:

\begin{equation} \label{eq_2_28}
	E^0_{\varphi}(r, z, t) = -\frac{\partial A^0_{\varphi}(r, z, t)}{\partial t}.
\end{equation}

Для перевода полученных результатов в трёхмерную задачу нужно перевести полученные значения вектора-потенциала $A^0_{\varphi}$ и напряженности электрического поля  $E^0_{\varphi}$ из цилиндрической в декартову систему координат. По формулам преобразования векторов (\ref{eq_2_29}) -- (\ref{eq_2_30}) найдем составляющие компоненты базисных вектор-функций уже для трёхмерной постановки задачи.

\begin{equation} \label{eq_2_29}
	\begin{cases}
		&	E_x^0(x, y, z, t) = -E_{\varphi}^0(\sqrt{x^2 + y^2}, z, t)\frac{y}{\sqrt{x^2 + y^2}},\\
		& 	E_y^0(x, y, z, t) = E_{\varphi}^0(\sqrt{x^2 + y^2}, z, t)\frac{x}{\sqrt{x^2 + y^2}},\\
		& 	E_z^0(x, y, z, t) = 0.
	\end{cases}
\end{equation}

\begin{equation} \label{eq_2_30}
	\begin{cases}
		&	A_x^0(x, y, z, t) = -A_{\varphi}^0(\sqrt{x^2 + y^2}, z, t)\frac{y}{\sqrt{x^2 + y^2}},\\
		& 	A_y^0(x, y, z, t) = A_{\varphi}^0(\sqrt{x^2 + y^2}, z, t)\frac{x}{\sqrt{x^2 + y^2}},\\
		& 	A_z^0(x, y, z, t) = 0.
	\end{cases}
\end{equation}

\section{Вариационная постановка трёхмерной задачи}

Принцип построения вариационного уравнения для трёхмерной задачи в целом похож на принцип построения вариационного уравнения для двумерной задачи \cite{8}, поэтому сразу получим слабую форму для (\ref{eq_1_6}): домножим обе части на пробную вектор-функцию $\overrightarrow{\Psi}$ и проинтегрируем по всей области $\Omega$.

\begin{equation} \label{eq_2_35}
	\int \limits_{\Omega} \frac{1}{\mu_0}  \text{rot} \left( \text{rot} \overrightarrow{\textbf{A}}^+ \right) \overrightarrow{\Psi} d \Omega + \int \limits_{\Omega} \sigma \frac{\partial \overrightarrow{\textbf{A}}^+}{\partial t} \overrightarrow{\Psi} d \Omega = \int \limits_{\Omega}(\sigma - \sigma_n) \overrightarrow{\textbf{E}^n} \overrightarrow{\Psi} d \Omega.
\end{equation}

Применяя формулу Грина и учитывая главные краевые условия для (\ref{eq_1_6}), в итоге получим:

\begin{equation} \label{eq_2_36}
	\int \limits_{\Omega} \frac{1}{\mu_0} \text{rot} \overrightarrow{\textbf{A}}^+ \text{rot}  \overrightarrow{\Psi} d \Omega + \int \limits_{\Omega} \sigma \frac{\partial \overrightarrow{\textbf{A}}^+}{\partial t} \overrightarrow{\Psi} d \Omega = \int \limits_{\Omega}(\sigma - \sigma_n) \overrightarrow{\textbf{E}}^n \overrightarrow{\Psi} d \Omega.
\end{equation}

\section{Конечноэлементная дискретизация трёхмерной задачи}

Сетку для решения задач на добавочное поле будем строить с помощью прямоугольных параллелепипедов. Будем сгущать сетку к аномальным элементам в расчётной области, чтобы не создавать лишних вычислительных затрат. 

Для решения задачи будем использовать билинейные базисные вектор-функции, которые задаются на параллелепипеде $\Omega_{rsp} = [x_p, x_{p+1}] \times [y_s, y_{s+1}] \times [z_p, z_{p+1}]$ следующим образом:

\begin{equation*}
	\overrightarrow{\psi}_1 = \left(
	\begin{array}{c}
		Y_1 \cdot Z_1\\
		0\\
		0\\
	\end{array}
	\right),
	\hspace{10mm}
	\overrightarrow{\psi}_2 = \left(
	\begin{array}{c}
		Y_2 \cdot Z_1\\
		0\\
		0\\
	\end{array}
	\right),
	\hspace{10mm}
	\overrightarrow{\psi}_3 = \left(
	\begin{array}{c}
		Y_1 \cdot Z_2\\
		0\\
		0\\
	\end{array}
	\right),
\end{equation*}

\begin{equation*}
	\overrightarrow{\psi}_4 = \left(
	\begin{array}{c}
		Y_2 \cdot Z_2\\
		0\\
		0\\
	\end{array}
	\right),
	\hspace{10mm}
	\overrightarrow{\psi}_5 = \left(
	\begin{array}{c}
		0\\
		X_1 \cdot Z_1\\
		0\\
	\end{array}
	\right),
	\hspace{10mm}
	\overrightarrow{\psi}_6 = \left(
	\begin{array}{c}
		0\\
		X_2 \cdot Z_1\\
		0\\
	\end{array}
	\right),
\end{equation*}


\begin{equation*}
	\overrightarrow{\psi}_7 = \left(
	\begin{array}{c}
		0\\
		X_1 \cdot Z_2\\
		0\\
	\end{array}
	\right),
	\hspace{10mm}
	\overrightarrow{\psi}_8 = \left(
	\begin{array}{c}
		0\\
		X_2 \cdot Z_2\\
		0\\
	\end{array}
	\right),
	\hspace{10mm}
	\overrightarrow{\psi}_9 = \left(
	\begin{array}{c}
		0\\
		0\\
		X_1 \cdot Y_1\\
	\end{array}
	\right),
\end{equation*}

\begin{equation*}
	\overrightarrow{\psi}_{10} = \left(
	\begin{array}{c}
		0\\
		0\\
		X_2 \cdot Y_1\\	\end{array}
	\right),
	\hspace{10mm}
	\overrightarrow{\psi}_{11} = \left(
	\begin{array}{c}
		0\\
		0\\
		X_1 \cdot Y_2\\\end{array}
	\right),
	\hspace{10mm}
	\overrightarrow{\psi}_{12} = \left(
	\begin{array}{c}
		0\\
		0\\
		X_2 \cdot Y_2\\
	\end{array}
	\right),
\end{equation*}
где

\begin{equation*} \label{eq_2_37}
	X_1(x) = \frac{x_{r + 1} - x}{x_{r + 1} - x_r}, \hspace{10mm} X_2(x) = \frac{x - x_r}{x_{r + 1} - x_r},
\end{equation*}

\begin{equation*} \label{eq_2_38}
	Y_1(y) = \frac{y_{s + 1} - y}{y_{s + 1} - y_s}, \hspace{10mm} Y_2(y) = \frac{y - y_s}{y_{s + 1} - y_s},
\end{equation*}

\begin{equation*} \label{eq_2_39}
	Z_1(z) = \frac{z_{p + 1} - z}{z_{p + 1} - z_p}, \hspace{10mm} Z_2(z) = \frac{z - z_p}{z_{p + 1} - z_p}.
\end{equation*}

\section{Построение матриц масс и жёсткости для трёхмерной задачи}

Формулы для вычисления глобальных матриц жёсткости $\textbf{G}$ и масс $\textbf{M}$ конечноэлементной СЛАУ имеют вид:

\begin{equation*} \label{eq_2_40}
	\hat{\textbf{G}}_{ij} = \int \limits_{\Omega} \frac{1}{\mu_0} \text{rot} \overrightarrow{\textbf{$\Psi_i$}} \cdot \text{rot} \overrightarrow{\textbf{$\Psi_j$}} d \Omega, \hspace{15mm} \hat{\textbf{M}}_{ij} = \int \limits_{\Omega} \sigma \overrightarrow{\textbf{$\Psi_i$}} \cdot \overrightarrow{\textbf{$\Psi_j$}} d \Omega.
\end{equation*}

Компоненты глобального вектора $\textbf{b}$ конечноэлементной СЛАУ определяются соотношением:

\begin{equation*} \label{eq_2_41}
	\textbf{b}_{i} = \int \limits_{\Omega} \overrightarrow{\textbf{F}} \cdot \overrightarrow{\textbf{$\Psi_i$}} d \Omega.
\end{equation*}

Локальная матрица жёсткости $\hat{\textbf{G}}$ на параллелепипеде при $\overline{\mu} = const$ принимает вид:

\begin{equation*}
	\hat{\textbf{G}} = \frac{1}{\overline{\mu}} \left(
	\begin{array}{ccc}
		\frac{h_x h_y}{6h_z}\textbf{G}_1 + \frac{h_x h_z}{6h_y}\textbf{G}_2 & -\frac{h_z}{6}\textbf{G}_2 & \frac{h_y}{6}\textbf{G}_3 \\
		-\frac{h_z}{6}\textbf{G}_2 & \frac{h_x h_y}{6h_z}\textbf{G}_1 + \frac{h_y h_z}{6h_x}\textbf{G}_2 & -\frac{h_x}{6}\textbf{G}_1 \\
		\frac{h_y}{6}\textbf{G}^\text{T}_3 & -\frac{h_x}{6}\textbf{G}_1 & \frac{h_x h_z}{6h_y}\textbf{G}_1 + \frac{h_y h_z}{6h_x}\textbf{G}_2 \\
	\end{array}
	\right),
\end{equation*}
где
\begin{equation*}
	\textbf{G}_1 = \left(
	\begin{array}{rrrr}
		2 & 1 & -2 & -1 \\
		1 & 2 & -1 & -2 \\
		-2 & -1 & 2 & 1 \\
		-1 & -2 & 1 & 2 \\
	\end{array}
	\right),
	\hspace{10mm}
	\textbf{G}_2 = \left(
	\begin{array}{rrrr}
		2 & -2 & 1 & -1 \\
		-2 & 2 & -1 & 2 \\
		1 & -1 & 2 & -2 \\
		-1 & 1 & -2 & 2 \\
	\end{array}
	\right),
\end{equation*}

\begin{equation*}
	\textbf{G}_3 = \left(
	\begin{array}{rrrr}
		-2 & 2 & -1 & 1 \\
		-1 & 1 & -2 & 2 \\
		2 & -2 & 1 & -1 \\
		1 & -1 & 2 & -2 \\
	\end{array}
	\right).
\end{equation*}

Матрицу $\hat{\textbf{M}}$ можно представить в виде:

\begin{equation*}
	\textbf{M} = \left(
	\begin{array}{ccc}
		\textbf{D} & \textbf{O} & \textbf{O}\\
		\textbf{O} & \textbf{D} & \textbf{O}\\
		\textbf{O} & \textbf{O} & \textbf{D} \\
	\end{array}
	\right) ,
\end{equation*}
где $\textbf{O}$ - матрица $4 \times 4$, состоящая из нулей подматрица, а $\textbf{D}$ определяется следующим образом:

\begin{equation*}
	\textbf{D} = \left(
	\begin{array}{rrrr}
		4 & 2 & 2 & 1 \\
		2 & 4 & 1 & 2 \\
		2 & 1 & 4 & 2 \\
		1 & 2 & 2 & 4 \\
	\end{array}
	\right).
\end{equation*}

Локальный вектор правой части определяется в виде:
\begin{equation} \label{eq_2_42}
	\hat{\textbf{b}}_i = \hat{\textbf{M}} \cdot \hat{\textbf{f}}_i,
\end{equation}
где $\hat{\textbf{f}}_i = \overrightarrow{\textbf{F}}(\hat{x}_c, \hat{y}_c, \hat{z}_c) \cdot \overrightarrow{\textbf{v}} / l$, для которого $\hat{x}_c, \hat{y}_c, \hat{z}_c$ -- координаты центра ребра, $\overrightarrow{\textbf{v}}$ -- вектор, направленный вдоль ребра $\Gamma$ и сонаправленный с базисной вектор-функцией $\overrightarrow{\textbf{$\Psi_i$}}$, $l$ -- длина вектора $\overrightarrow{\textbf{F}}$.

Применяя трехслойную неявную схему аппроксимации по времени, получим итоговое СЛАУ для вариационного уравнения (\ref{eq_2_36}):

\begin{equation} \label{eq_2_43}
	\left(\textbf{G} + \tau_0 \textbf{M}^{\sigma}\right) \textbf{q} = \textbf{M}^{\sigma - \sigma_n} \cdot \textbf{E} + \tau_1 \textbf{M}^{\sigma}  \textbf{q}^{\Leftarrow 1} - \tau_2 \textbf{M}^{\sigma}  \textbf{q}^{\Leftarrow 2}.
\end{equation}

% Далее глава пркатической части.
\chapter{Практическая часть}

\section{Генерация трёхмерной сетки с ячейками в виде шестигранников}

При написании программы был использован следующий подход к построению сетки на шестигранных элементах.\\
\texttt{\hspace*{0.5em}1. [lines amount x] 2 [lines amount y] 2 [lines amount z] 2\\}
\texttt{\hspace*{0.5em}2. [field description of points]\\}
\texttt{\hspace*{0.5em}3. 0.0 0.0 0.0 \qquad 1.0 0.0 0.0\\}
\texttt{\hspace*{0.5em}4. 0.0 1.0 0.0 \qquad 1.0 1.0 0.0\\}
\texttt{\hspace*{0.5em}5. 0.0 0.0 1.0 \qquad 1.0 0.0 1.0\\}
\texttt{\hspace*{0.5em}6. 0.0 1.0 1.0 \qquad 1.0 1.0 1.0\\}	
\texttt{\hspace*{0.5em}7. [unique areas amount] 1\\}
\texttt{\hspace*{0.5em}8. [unique areas description]\\}
\texttt{\hspace*{0.5em}9. 1 0 1 0 1 0 1\\}
\texttt{10. [unique areas coefficients description]\\}
\texttt{11. 1 1.0 1.0\\}
\texttt{12. [delimiters above X description] 1 1.0\\} 
\texttt{13. [delimiters above Y description] 3 1.1\\}
\texttt{14. [delimiters above Z description] 4 0.8\\}
\texttt{15. [borders amount] 6\\}
\texttt{16. [borders description]\\}
\texttt{17. 1 1 0 1 0 0 0 1\\}
\texttt{18. 1 1 0 1 1 1 0 1\\}
\texttt{19. 1 1 0 0 0 1 0 1\\}
\texttt{20. 1 1 1 1 0 1 0 1\\}
\texttt{21. 1 1 0 1 0 1 0 0\\}
\texttt{22. 1 1 0 1 0 1 1 1\\}

В первой строке заданы количество опорных узлов $N_x^W, \,N_y^W, \,N_z^W,$ базовой плоскости по осям $X,\,Y,\,Z$ соответственно. С третьей по шестую строки перечисленны тройки чисел $\left(x_i, \, y_i, \, z_i\right)$ - как раз и определяющие эти опорные узлы. 

В седьмой строке указано количество уникальных областей в расчётной области, которые имеют определённые уникальные значения физических параметров $\mu$ и $\sigma$. Начиная с девятой строки (в общем случае должен быть построчный перечень каждой области) описывается геометрическое расположение $i$ - ой области.  В одиннадцатой строке указаны уникальные значения параметров $\mu$ и $\sigma$ для $i$ - ой области.

В строках с двенадцатой по четырнадцатую описывается количество и характер необходимых разбиений для осей $X,\,Y,\,Z$ соответственно.

В пятнадцатой строке целочисленным значением задаётся количество границ. Далее с семнадцатой по двадцать вторую строки описывается расположение и характер этих границ. Первым числом задаётся тип краевого условия (т.е. принимает значения 1 или 2), вторым числом задаётся номер формулы, третьим первая координатная линия по оси $X$, четвёртым вторая координатная линия по оси $X$, пятым и шестым аналогично по оси $Y$ и седьмым и восьмым по оси $Z$.

Пример расчётной области этой фигуры изображён на рисунке (\ref{fig:ExampleCube}).

\begin{figure}
	\centering
	\vspace*{0.7cm}
	\includegraphics[width=0.7\linewidth]{images/ExampleCube.png}
	\caption{Расчетная область для кубика}
	\label{fig:ExampleCube}
\end{figure}

Попробуем подробить расчётную область (\ref{fig:ExampleCube}) на несколько частей. Получим сетку изображённую на рисунке (\ref{fig:GridCube}).

\begin{figure}
	\centering
	\vspace*{0.7cm}
	\includegraphics[width=0.7\linewidth]{images/GridCube.png}
	\caption{Секта для кубика}
	\label{fig:GridCube}
\end{figure}

Приведём ещё насколько примеров для построения сеток на шестигранниках, изображённых на рисунках  \ref{fig:Emerald} - \ref{fig:DEmerald}.

\begin{figure}
	\begin{minipage}[h]{0.49\linewidth}
		\center{\includegraphics[width=0.5\linewidth]{images/EmeraldField.png} \\ а)}
	\end{minipage}
	\hfill
	\begin{minipage}[h]{0.49\linewidth}
		\center{\includegraphics[width=0.5\linewidth]{images/EmeraldMesh.png} \\ б)}
	\end{minipage}
	\caption{Расчётная область в форме изумруда (а) и сетка к ней (б).}
	\label{fig:Emerald}
\end{figure}

\begin{figure}
	\begin{minipage}[h]{0.49\linewidth}
		\center{\includegraphics[width=0.5\linewidth]{images/PyramidField.png} \\ а)}
	\end{minipage}
	\hfill
	\begin{minipage}[h]{0.49\linewidth}
		\center{\includegraphics[width=0.5\linewidth]{images/PyramidMesh.png} \\ б)}
	\end{minipage}
	\caption{Расчётная область в форме скошенной пирамиды (а) и сетка к ней (б).}
	\label{fig:Pyramid}
\end{figure}

\begin{figure}
	\begin{minipage}[h]{0.49\linewidth}
		\center{\includegraphics[width=0.5\linewidth]{images/HGField.png} \\ а)}
	\end{minipage}
	\hfill
	\begin{minipage}[h]{0.49\linewidth}
		\center{\includegraphics[width=0.5\linewidth]{images/HGMesh.png} \\ б)}
	\end{minipage}
	\caption{Расчётная область в форме песочных часов (а) и сетка к ней (б).}
	\label{fig:HG}
\end{figure}

\begin{figure}
	\begin{minipage}[h]{0.49\linewidth}
		\center{\includegraphics[width=0.5\linewidth]{images/BathField.png} \\ а)}
	\end{minipage}
	\hfill
	\begin{minipage}[h]{0.49\linewidth}
		\center{\includegraphics[width=0.5\linewidth]{images/BathMesh.png} \\ б)}
	\end{minipage}
	\caption{Расчётная область в форме ванной (а) и сетка к ней (б).}
	\label{fig:Bath}
\end{figure}

\begin{figure}
	\begin{minipage}[h]{0.49\linewidth}
		\center{\includegraphics[width=0.5\linewidth]{images/DEmeraldField.png} \\ а)}
	\end{minipage}
	\hfill
	\begin{minipage}[h]{0.49\linewidth}
		\center{\includegraphics[width=0.5\linewidth]{images/DEmeraldMesh.png} \\ б)}
	\end{minipage}
	\caption{Расчётная область в форме детализированного изумруда (а) и сетка к ней (б).}
	\label{fig:DEmerald}
\end{figure}

Таким образом, программа для построения сетки может строить достаточно геометрически сложные фигуры.

\section{Численное интегрирование}

При расчёте элементов локальных матриц жёсткости (\ref{eq_1_11}) и масс (\ref{eq_1_12}) будем использовать численное интегрирование методами Гаусса разных порядков (2, 3, 4, 5). Результаты численного интегрирования на некоторых функциях приведены в таблицах \ref{tab:numIntegr1} - \ref{tab:numIntegr7}. Область интегрирования для всех функций единый: $\Omega_E \in \left[-1; 1\right]_x \cross \left[-1; 1\right]_y \cross \left[-1; 1\right]_z$.

\begin{table}
	\caption{Тестирование численного интегрирования на функции $u = 2$.}
	\centering
	\small
	\begin{tabularx}{1.0\textwidth}{| >{\raggedright\arraybackslash}X | >{\raggedright\arraybackslash}X | >{\raggedright\arraybackslash}X |>{\raggedright\arraybackslash}X |>{\raggedright\arraybackslash}X |}
		\hline
		\centering{Аналитический результат} & \centering{Гаусс 2} & \centering{Гаусс 3} & \centering{Гаусс 4} & \centering{Гаусс 5} \tabularnewline \hline
		
		\centering{16.0} & \centering{1.6000000e+01}& \centering{1.6000000e+01} & \centering{1.6000000e+01} & \centering{1.6000000e+01} \tabularnewline \hline
		
	\end{tabularx}
	\label{tab:numIntegr1}
\end{table}


\begin{table}
	\caption{Тестирование численного интегрирования на функции $u = x + y + z$.}
	\centering
	\small
	\begin{tabularx}{1.0\textwidth}{| >{\raggedright\arraybackslash}X | >{\raggedright\arraybackslash}X | >{\raggedright\arraybackslash}X |>{\raggedright\arraybackslash}X |>{\raggedright\arraybackslash}X |}
		\hline
		\centering{Аналитический результат} & \centering{Гаусс 2} & \centering{Гаусс 3} & \centering{Гаусс 4} & \centering{Гаусс 5} \tabularnewline \hline
		
		\centering{0.0} & \centering{0.0000000e+00}& \centering{-2.2204460e-16} & \centering{5.6898930e-16} & \centering{-6.5225603e-16} \tabularnewline \hline
		
	\end{tabularx}
	\label{tab:numIntegr2}
\end{table}

\begin{table}
	\caption{Тестирование численного интегрирования на функции $u = x^2 + y^2 + z^2$.}
	\centering
	\small
	\begin{tabularx}{1.0\textwidth}{| >{\raggedright\arraybackslash}X | >{\raggedright\arraybackslash}X | >{\raggedright\arraybackslash}X |>{\raggedright\arraybackslash}X |>{\raggedright\arraybackslash}X |}
		\hline
		\centering{Аналитический результат} & \centering{Гаусс 2} & \centering{Гаусс 3} & \centering{Гаусс 4} & \centering{Гаусс 5} \tabularnewline \hline
		
		\centering{8.0} & \centering{8.0000000e+00}& \centering{8.0000000e+00} & \centering{8.0000000e+00} & \centering{8.0000000e+00} \tabularnewline \hline
		
	\end{tabularx}
	\label{tab:numIntegr3}
\end{table}

\begin{table}
	\caption{Тестирование численного интегрирования на функции $u = x \cdot y \cdot z$.}
	\centering
	\small
	\begin{tabularx}{1.0\textwidth}{| >{\raggedright\arraybackslash}X | >{\raggedright\arraybackslash}X | >{\raggedright\arraybackslash}X |>{\raggedright\arraybackslash}X |>{\raggedright\arraybackslash}X |}
		\hline
		\centering{Аналитический результат} & \centering{Гаусс 2} & \centering{Гаусс 3} & \centering{Гаусс 4} & \centering{Гаусс 5} \tabularnewline \hline
		
		\centering{0.0} & \centering{8.0000000e+00}& \centering{0.0000000e+00} & \centering{0.0000000e+00} & \centering{8.6736174e-18} \tabularnewline \hline
		
	\end{tabularx}
	\label{tab:numIntegr4}
\end{table}

\begin{table}
	\caption{Тестирование численного интегрирования на функции $u = x^2 \cdot y^2 \cdot z^2$.}
	\centering
	\small
	\begin{tabularx}{1.0\textwidth}{| >{\raggedright\arraybackslash}X | >{\raggedright\arraybackslash}X | >{\raggedright\arraybackslash}X |>{\raggedright\arraybackslash}X |>{\raggedright\arraybackslash}X |}
		\hline
		\centering{Аналитический результат} & \centering{Гаусс 2} & \centering{Гаусс 3} & \centering{Гаусс 4} & \centering{Гаусс 5} \tabularnewline \hline
		
		\centering{$\frac{8}{27}$ $\approx$ 0.29630} & \centering{2.9629630e-01}& \centering{2.9629630e-01} & \centering{2.9629630e-01} & \centering{2.9629630e-01} \tabularnewline \hline
		
	\end{tabularx}
	\label{tab:numIntegr5}
\end{table}

\begin{table}
	\caption{Тестирование численного интегрирования на функции $u = \text{cos}(x + y + z)$.}
	\centering
	\small
	\begin{tabularx}{1.0\textwidth}{| >{\raggedright\arraybackslash}X | >{\raggedright\arraybackslash}X | >{\raggedright\arraybackslash}X |>{\raggedright\arraybackslash}X |>{\raggedright\arraybackslash}X |}
		\hline
		\centering{Аналитический результат} & \centering{Гаусс 2} & \centering{Гаусс 3} & \centering{Гаусс 4} & \centering{Гаусс 5} \tabularnewline \hline
		
		\centering{4.7666...} & \centering{4.7063579e+00}& \centering{4.7671091e+00} & \centering{4.7665835e+00} & \centering{4.7665859e+00} \tabularnewline \hline
		
	\end{tabularx}
	\label{tab:numIntegr6}
\end{table}

\begin{table}
	\caption{Тестирование численного интегрирования на функции $u = e^{x + y + z}$.}
	\centering
	\small
	\begin{tabularx}{1.0\textwidth}{| >{\raggedright\arraybackslash}X | >{\raggedright\arraybackslash}X | >{\raggedright\arraybackslash}X |>{\raggedright\arraybackslash}X |>{\raggedright\arraybackslash}X |}
		\hline
		\centering{Аналитический результат} & \centering{Гаусс 2} & \centering{Гаусс 3} & \centering{Гаусс 4} & \centering{Гаусс 5} \tabularnewline \hline
		
		\centering{12.9845...} & \centering{1.2857243e+01}& \centering{1.2983458e+01} & \centering{1.2984538e+01} & \centering{1.2984543e+01} \tabularnewline \hline
		
	\end{tabularx}
	\label{tab:numIntegr7}
\end{table}

\section{Процесс построения локальных матриц жёсткости и масс}

Рассмотрим процесс построения локальных матриц жёсткости и масс. При генерации локальной матрицы масс на параллелепипеде в векторном МКЭ используется следующая локальная матрица

\begin{equation} \label{eq_2_1}
	\begin{gathered}
		\hat{\textbf{M}} = \gamma \frac{h_x h_y h_z}{36} 
		\begin{pmatrix} 
			\textbf{D} & \textbf{0} & \textbf{0}\\
			\textbf{0} & \textbf{D} & \textbf{0}\\
			\textbf{0} & \textbf{0} & \textbf{D}\\
		\end{pmatrix},
	\end{gathered}
\end{equation}
где $\textbf{0}$ - матрица размером 4 $\cross$ 4, полностью заполненная нулями, а

\begin{equation*}
	\begin{gathered}
		\hat{\textbf{D}} =
		\begin{pmatrix}
			4 & 2 & 2 & 1\\
			2 & 4 & 1 & 2\\
			2 & 1 & 4 & 2\\
			1 & 2 & 2 & 4\\
		\end{pmatrix}.
	\end{gathered}
\end{equation*}

Тогда на единичном элементе $\Omega_E \in \left[-1; 1\right]_x \cross \left[-1; 1\right]_y \cross \left[-1; 1\right]_z$ и при $\gamma = 1$ матрица (\ref{eq_2_1}) примет вид:


\begin{equation*}
	\begin{gathered}
		\hat{\textbf{M}} = \frac{2}{9} 
		\begin{pmatrix} 
			\textbf{D} & \textbf{0} & \textbf{0}\\
			\textbf{0} & \textbf{D} & \textbf{0}\\
			\textbf{0} & \textbf{0} & \textbf{D}\\
		\end{pmatrix}.
	\end{gathered}
\end{equation*}

Попробуем сделать ту же самую процедуру при $\gamma = 1$ и на элемента $\Omega_E \in \left[-1; 1\right]_x \cross \left[-1; 1\right]_y \cross \left[-1; 1\right]_z$, но используя интегралы из (\ref{eq_1_12}). Элементы сгенерированной матрицы в виде консольного вывода программы изображены на рисунке \ref{fig:GeneratedMatrixMass}.

\begin{figure}
	\centering
	\vspace*{0.7cm}
	\includegraphics[width=1.0\linewidth]{images/M.png}
	\caption{Консольный вывод сгенерированной матрицы масс на $\Omega_E$}
	\label{fig:GeneratedMatrixMass}
\end{figure}

Локальная матрица жёсткости $\hat{\textbf{G}}$ на параллелепипеде $\Omega_E \in \left[-1, 1\right]_x \cross \left[-1, 1\right]_y \cross \left[-1, 1\right]_z$ при $\overline{\mu} = 1$ принимает вид:

\begin{equation*}
	\hat{\textbf{G}} = \left(
	\begin{array}{ccc}
		\frac{1}{3}\textbf{G}_1 + \frac{1}{3}\textbf{G}_2 & -\frac{1}{3}\textbf{G}_2 & \frac{1}{3}\textbf{G}_3 \\
		-\frac{1}{3}\textbf{G}_2 & \frac{1}{3}\textbf{G}_1 + \frac{1}{3}\textbf{G}_2 & -\frac{1}{3}\textbf{G}_1 \\
		\frac{1}{3}\textbf{G}^\text{T}_3 & -\frac{1}{3}\textbf{G}_1 & \frac{1}{3}\textbf{G}_1 + \frac{1}{3}\textbf{G}_2 \\
	\end{array}
	\right),
\end{equation*}
где
\begin{equation*}
	\textbf{G}_1 = \left(
	\begin{array}{rrrr}
		2 & 1 & -2 & -1 \\
		1 & 2 & -1 & -2 \\
		-2 & -1 & 2 & 1 \\
		-1 & -2 & 1 & 2 \\
	\end{array}
	\right),
	\hspace{10mm}
	\textbf{G}_2 = \left(
	\begin{array}{rrrr}
		2 & -2 & 1 & -1 \\
		-2 & 2 & -1 & 2 \\
		1 & -1 & 2 & -2 \\
		-1 & 1 & -2 & 2 \\
	\end{array}
	\right),
\end{equation*}

\begin{equation*}
	\textbf{G}_3 = \left(
	\begin{array}{rrrr}
		-2 & 2 & -1 & 1 \\
		-1 & 1 & -2 & 2 \\
		2 & -2 & 1 & -1 \\
		1 & -1 & 2 & -2 \\
	\end{array}
	\right).
\end{equation*}

Теперь, проверим генерацию матрицы жёсткости через формулу (\ref{eq_1_11}). Консольный вывод содержимого матрицы указан на рисунке \ref{fig:GeneratedMatrixStiffness}

\begin{figure}
	\centering
	\vspace*{0.7cm}
	\includegraphics[width=1.0\linewidth]{images/G.png}
	\caption{Консольный вывод сгенерированной матрицы масс на $\Omega_E$}
	\label{fig:GeneratedMatrixStiffness}
\end{figure}

\section{Решение СЛАУ}

Через LOS или Pardiso.





%\textbf{Здесь будет программная реализация $\downarrow$}


\chapter{Исследования}

\section{Описание исследований и расположение приёмников}

Для начала будем проводить исследования в горизонтально-слоистой среде без каких-либо аномалий. Напомню, что на рисунке \ref{fig:example} изображен срез горизонтально-слоистой среды, который мы используем в качестве решения нормального поля на первом этапе разделения полей. Далее рассмотрим горизонтально-слоистую среду с двумя аномалиями вместе, т.е для решения задачи на добавочное поле будем использовать сетку, учитывающую сразу две аномалии. После этого последовательно рассмотрим добавление в среду сначала первой аномалии потом второй, а затем сначала второй и после первой. На заключительном этапе рассмотрим временные затраты на решение задачи при использовании многоэтапной схемы разделения поля и при разбиении поля на нормальное и добавочное.

Пусть источник индукционного поля имеет радиус $R = 100$ м. от оси симметрии и имееет силу тока, равную $J_{\varphi} = 1.0$ А. Также условимся, что источник работал достаточно долго, чтобы создать стабильное электромагнитное поле. Сетка по времени: $t \in [0.0; 1.0]$ на 100 временных слоёв c начальным шагом $h_t = 10^{-5}$ с. и коэффициентом разрядки $t_k = 1.1$. После отключим наш источник, т.е. $J_{\varphi} = 0.0$ A при $t > 0.0$. Возьмём 4 приемника и расположим их вдоль линии $x = 0$ между аномалиями. Пусть они будут располагаться на расстояниях 101 м., 1000 м., 2000 м., 3000 м. от центра симметрии. На рисунке \ref{fig:receivers_info} тёмно-бирюзовой кривой нарисован индукционный источник тока, контурными линиями нарисованы положение аномальных объектов в среде, точками -- расположение приёмников.


\begin{figure}
	\centering
	\vspace*{0.7cm}
	\includegraphics[width=0.7\linewidth]{images/receivers_location.png}
	\caption{Расположение приёмников}
	\label{fig:receivers_info}
\end{figure}


\section{Исследование нормального поля без аномалий}

На рисунках \ref{fig:E_phi_0} -- \ref{fig:E_phi_3} представлено распространение этого поля в среде в начальный, промежуточные и последний моменты времени. Также, проведём замеры значения напряжённости электрического поля в них. Полученные зависимости $E^0$ от времени изображены на рисунке \ref{fig:LogE}.

\begin{figure}
	\centering
	\vspace*{0.7cm}
	\includegraphics[width=0.95\linewidth]{images/Answer_E_time_layer_1.png}
	\caption{Решение $E_{\varphi}$ при $t = 10^{-5}$c}
	\label{fig:E_phi_0}
\end{figure}

\begin{figure}
	\centering
	\vspace*{0.7cm}
	\includegraphics[width=0.95\linewidth]{images/Answer_E_time_layer_15.png}
	\caption{Решение $E_{\varphi}$ при $t = 0.01$c}
	\label{fig:E_phi_1}
\end{figure} 

\begin{figure}
	\centering
	\vspace*{0.7cm}
	\includegraphics[width=0.95\linewidth]{images/Answer_E_time_layer_45.png}
	\caption{Решение $E_{\varphi}$ при $t = 0.1$c}
	\label{fig:E_phi_2}
\end{figure} 

\begin{figure}
	\centering
	\vspace*{0.7cm}
	\includegraphics[width=0.95\linewidth]{images/Answer_E_time_layer_60.png}
	\caption{Решение $E_{\varphi}$ при $t = 1$c}
	\label{fig:E_phi_3}
\end{figure} 

\begin{figure}
	\centering
	\vspace*{0.7cm}
	\includegraphics[width=1.0\linewidth]{images/Log_E.png}
	\caption{Зависимость значения модуля $E^0$ от времени в разных приёмниках}
	\label{fig:LogE}
\end{figure} 

Как видим, значения электрической напряжённости поля на приёмниках не имеют каких-либо резких колебаний и постепенно снижаются. Из этого можно заключить, что, как и предполагалось, никаких аномальных зон в исследуемой области нет. 

\section{Исследование в среде с двумя аномалиями}

Теперь расположим в расчётной области два аномальных объекта. Пусть первый объект имеет удельную электропроводность $\sigma_1 = 5.0$ См/м и находится в $\Omega_1 \in [2500, 4000]_x \times [500, 1250]_y \times [-2000, -750]_z$, а второй $\sigma_2 = 10.0$ См/м и находится в $\Omega_2 \in [2500, 4000]_x \times [-1250, -500]_y \times [-2000, -750]_z$. Их расположение в сечениях изображено на рисунках \ref{fig:Anomaly1} -- \ref{fig:Anomaly2}.  

\begin{figure}
	\centering
	\vspace*{0.7cm}
	\includegraphics[width=0.8\linewidth]{images/Anomalies_1.png}
	\caption{Срез среды при $x = 2500$м}
	\label{fig:Anomaly1}
\end{figure}


\begin{figure}
	\centering
	\vspace*{0.7cm}
	\includegraphics[width=0.7\linewidth]{images/Anomalies_2.png}
	\caption{Срез среды при $z = -1000$м}
	\label{fig:Anomaly2}
\end{figure}

Тогда графики зависимости значения напряжённости электрического поля $E$ от времени на выбранных нами приёмниках представлены на рисунке \ref{fig:LogE_both}. Процесс рассматривался с 0.002 секунды. 

\begin{figure}
	\centering
	\vspace*{0.7cm}
	\includegraphics[width=1.0\linewidth]{images/Log_E_both.png}
	\caption{Зависимость значения модуля $E^+$ от времени в приёмниках при одновременном разделении аномалий}
	\label{fig:LogE_both}
\end{figure} 

Как можно заметить, со временем, значение модуля напряжённости электрического поля добавочных объектов сначала увеличивается, а затем уменьшается. Объяснить это можно тем, что рассеиваемое поле горизонтально-слоистой среды со временем начинает оказывать влияние на аномальный по электропроводности объект. Из-за этого модуль значения напряжённости электромагнитного поля сначала растёт, а затем плавно также рассеивается вместе с полем горизонтально-слоистой среды.

\section{Исследование многоэтапного разделения полей}

Теперь рассмотрим постепенное добавление аномальных зон в расчётную область. Пусть на первом этапе, в качестве добавочного поля будем считать поле, создаваемое объектом с удельной электропроводностью $\sigma$ = 5.0 См/м. На втором этапе в качестве нормального поля будем рассматривать поле, полученное на предыдущем этапе, а в качестве добавочного, поле создаваемое объектом с электропроводностью $\sigma$ = 10.0 См/м. Графики изменения значения напряжённости электрического поля от времени после добавления первой аномалии, а затем второй изображёны на рисунке \ref{fig:LogE_1_plus_2}.

\begin{figure}
	\centering
	\vspace*{0.7cm}
	\includegraphics[width=0.75\linewidth]{images/Log_E_1_plus_2.png}
	\caption{Зависимость значения модуля $E^+$ от времени в разных приёмниках при разделении сначала первой, а затем второй аномалии}
	\label{fig:LogE_1_plus_2}
\end{figure} 


По началу значения напряжённости поля повышаются. Резкий скачок вниз можно объяснить инверсией электромагнитного поля --  направление распространения может измениться на противоположное по знаку значение.

Теперь рассмотрим постепенное добавление аномальных зон в расчётную область наоборот. На первом этапе, в качестве добавочного поля будем считать поле, создаваемое объектом с удельной электропроводностью $\sigma$ = 10.0 См/м. На втором этапе в качестве нормального поля будем рассматривать поле, полученное на предыдущем этапе, а в качестве добавочного, поле создаваемое объектом с электропроводностью $\sigma$ = 5.0 См/м. График изменения значения напряжённости электрического поля от времени после добавления первой аномалии, а затем второй изображён на рисунке \ref{fig:LogE_2_plus_1}.

\begin{figure}
	\centering
	\vspace*{0.7cm}
	\includegraphics[width=0.75\linewidth]{images/Log_E_2_plus_1.png}
	\caption{Зависимость значения модуля $E^+$ от времени в разных приёмниках при разделении сначала второй, а затем первой аномалии}
	\label{fig:LogE_2_plus_1}
\end{figure} 

На рисунке \ref{fig:LogE_compare_1_and_2} изображено сравнение результатов первого и второго подхода к многоэтапному разделению полей. Цветными линиями показано значение напряжённости электрического поля при разделении сначала второй, а затем первой, чёрным пунктиром значение напряжённости при разделении сначала первой, а затем второй аномалий.

\begin{figure}
	\centering
	\vspace*{0.7cm}
	\includegraphics[width=1.0\linewidth]{images/Log_E_compare_1_and_2.png}
	\caption{Сравнение значений модуля $E^+$ от времени на разных приёмниках при разной очерёдности разделении аномалий}
	\label{fig:LogE_compare_1_and_2}
\end{figure} 


Как выяснилось, очерёдность разделения полей не дала видимых различий, следовательно результат не зависит от очерёдности добавления аномалий в расчётную область.

Поскольку очерёдность добавления аномальных объектов в расчётную область не даёт очевидной разницы, то рассмотрим на рисунке \ref{fig:LogE_compare_both_and_msrp} сравнение значений $E^+$ при одновременном разделении сразу двух объектов и при использовании многоэтапной схемы, сначала выделяя первый, а затем второй объект. Цветными линиями показаны значения при использовании многоэтапной схемы разделения полей, пунктиром при одновременном разделении сразу двух аномалий.

\begin{figure}
	\centering
	\vspace*{0.7cm}
	\includegraphics[width=1.0\linewidth]{images/Log_E_compare_both_with_msrp.png}
	\caption{Сравнение значений модуля $E^+$ от времени в разных приёмниках при одноэтапном и многоэтапном разделении}
	\label{fig:LogE_compare_both_and_msrp}
\end{figure}  
 

\section{Исследование явления взаимоиндукции}

Теперь предположим, что при использовании многоэтапной схемы разделения полей для обоих наших объектов можно найти, не учитывая влияния другого объекта, а учитывая лишь влияние поля в горизонтально-слоистой среде. Иными словами, значение $\overrightarrow{\textbf{E}}^n$ при учёте второй по очереди аномалии в уравнении (\ref{eq_1_6}) будет рассматриваться не суммой $E^0 + E^a$ как мы делали это прежде, а просто через $E^0$, где $E^0$ -- напряжённость электрического поля от влияния горизонтально-слоистой среды, а $E^a$ -- напряжённость электрического поля от влияния первой аномалии. На рисунках \ref{fig:LogE_isolated_1} -- \ref{fig:LogE_isolated_4} приведены сравнения значений для разных приёмников. Сплошной линией отображена зависимость при учёте напряжённости электрического поля только горизонтально-слоистой среды (то есть просто $E^0$), а пунктиром с учётом ещё и влияния другой аномалии (то есть $E^0 + E^a$).

\begin{figure}
	\centering
	\vspace*{0.7cm}
	\includegraphics[width=0.7\linewidth]{images/Log_E_compare_3000.png}
	\caption{Сравнение аномалии от двух объектов с суммой аномалий от одиночных объектов на приёмнике удалённом на 3000 м}
	\label{fig:LogE_isolated_1}
\end{figure} 

\begin{figure}
	\centering
	\vspace*{0.7cm}
	\includegraphics[width=0.7\linewidth]{images/Log_E_compare_2000.png}
	\caption{Сравнение аномалии от двух объектов с суммой аномалий от одиночных объектов на приёмнике удалённом на 2000 м}
	\label{fig:LogE_isolated_2}
\end{figure} 

\begin{figure}
	\centering
	\vspace*{0.7cm}
	\includegraphics[width=0.7\linewidth]{images/Log_E_compare_1000.png}
	\caption{Сравнение аномалии от двух объектов с суммой аномалий от одиночных объектов на приёмнике удалённом на 1000 м}
	\label{fig:LogE_isolated_3}
\end{figure} 

\begin{figure}
	\centering
	\vspace*{0.7cm}
	\includegraphics[width=0.7\linewidth]{images/Log_E_compare_101.png}
	\caption{Сравнение аномалии от двух объектов с суммой аномалий от одиночных объектов на приёмнике удалённом на 101 м}
	\label{fig:LogE_isolated_4}
\end{figure} 

На рисунках \ref{fig:LogE_isolated_1} -- \ref{fig:LogE_isolated_3} отчётливо видно, что пунктирная прямая, начиная примерно с 0.01 секунды, несколько выше, чем сплошная. Такую разницу можно объяснить явлением взаимоиндукции электромагнитных полей. Соответственно, такой учёт аномальных объектов при многоэтапном разделении полей не стоит применять если аномальные объекты находятся достаточно близко друг к другу.
\chapter*{Заключение}

\addcontentsline{toc}{chapter}{Заключение}

При выполнений курсовой работы по теме  "Разработка программы для моделирования трехмерного электромагнитного поля на шестигранниках с использованием векторного МКЭ." было разработано программное обеспечение, позволяющее строить сетку достаточно сложных геометрических фигур, используя трёхмерные "шестигранники" в качестве конечных элементов. Также разработано программное обеспечение, позволяющее вычислять локальные матрицы масс и жёсткости, а также значения локального вектора правой части на шестигранниках. 


\newpage

\addcontentsline{toc}{chapter}{Список используемых источников}
\renewcommand\bibname{СПИСОК ИСПОЛЬЗУЕМЫХ ИСТОЧНИКОВ}

\begin{thebibliography}{00}
	\bibitem{1}
			М.С. Жданов Электроразведка. - М.: Недра, 1986. - 316 с.
    \bibitem{2}
			А.А. Логачев, В.П. Захаров Магниторазведка. - 5 изд. - Ленинград: Недра, 1979. - 350 с.
    \bibitem{3}
    		Pavel Solin Partial Differential Equations and the Finite Element Method. - Hoboken, New Jersey: A JOHN WILEY $\&$ SONS, INC., 2006.
    \bibitem{4}
    		А.Н. Тихонов, А.А. Самарский Уравнения математической физики: Учеб.пособие. / А.Н. Тихонов, А.А. Самарский — 6-е изд., — М: Изд-во МГУ, 1999 — 799 с.
   	\bibitem{5}		
 			М.Г. Персова, Ю.Г. Соловейчик, М.Г. Токарева, М.В. Абрамов 3D-моделирование процессов индукционной вызванной поляризации при возбуждении токовой петлей и проблема эквивалентности // Научный вестник НГТУ. - 2013. - №2(51). - С. 53 - 61.
    \bibitem{6}
   			М. Г. Персова, Ю. Г. Соловейчик, Г. М. Тригубович, М. В. Абрамов, А. А. Заборцева О вычислении трёхмерного нестационарного поля вертикальной электрической линии в удалённой обсаженной скважине // Сибирский журнал индустриальной математики. - 2007. - №3(31). - С. 114 - 127.
    \bibitem{7}
			Ю.Г. Соловейчик, М.Э. Рояк, М.Г. Персова Метод конечных элементов для скалярных и векторных задач Учеб. пособие. — Новосибирск: Изд-во НГТУ, 2007 — 896 с.
	\bibitem{8}
			М.Ю.Баландин, Э.П.Шурина Векторный метод конечных элементов: Учеб. пособие. - Новосибирск: Изд-во НГТУ, 2001 — 69 с.
	\bibitem{9}
  			М.Ю.Баландин, Э.П.Шурина Методы решения СЛАУ большой размерности: Учеб. пособие. - Новосибирск: Изд-во НГТУ, 2000 — 70 с.
	\bibitem{10}
    		М.Г. Персова, Ю.Г. Соловейчик, Д.В. Вагин, П.А. Домников, Ю.И. Кошкина Численные методы в уравнениях математической физики.  - Новосибирск: Изд-во НГТУ, 2016 — 60 с.
    \bibitem{11}
    		Вагин Денис Владимирович Разработка методов конечноэлементного моделирования трехмерных электромагнитных полей на неструктурированных сетках: автореф. дис. канд. техн. наук: 05.13.18. - Новосибирск, 2012.
    \bibitem{12}	
    		Тракимус Юрий Викторович Разработка и применение схем конечноэлементного моделирования электромагнитных полей в задачах электроразведки с использованием скважин: автореф. дис. канд. техн. наук: 05.13.18. - Новосибирск, 2007.
    \bibitem{13}
    		П.А. Домников Решение систем конечноэлементных уравнений при моделировании гармонических геоэлектромагнитных полей в трехмерных задачах морской электроразведки // Доклады Академии наук высшей школы Российской Федерации. - 2013. - №1 (20).
\end{thebibliography}

\newpage
%\input{tex/appendix2}
%\input{tex/appendix3}
\chapter*{Приложение 3. Текст программы}
\addcontentsline{toc}{chapter}{Приложение 3. Текст программы}
\label{code: code}
\subsection*{Program.cs}
\lst{cs}{code/Program.cs}

\subsection*{LocalMatrix.cs}
\lst{cs}{code/LocalMatrix.cs}

\subsection*{LocalVector.cs}
\lst{cs}{code/LocalVector.cs}

\subsection*{FEM.cs}
\lst{cs}{code/FEM.cs}

\subsection*{FEM2D.cs}
\lst{cs}{code/FEM2D.cs}

\subsection*{FEM3D.cs}
\lst{cs}{code/FEM3D.cs}

\subsection*{Mesh.cs}
\lst{cs}{code/Mesh.cs}

\subsection*{Mesh2Dim.cs}
\lst{cs}{code/Mesh2Dim.cs}

\subsection*{Mesh3Dim.cs}
\lst{cs}{code/Mesh3Dim.cs}

\subsection*{LocalMatrix3D.cs}
\lst{cs}{code/LocalMatrix3D.cs}

\subsection*{LocalVector3D.cs}
\lst{cs}{code/LocalVector3D.cs}

\subsection*{MathOperations.cs}
\lst{cs}{code/MathOperations.cs}

\subsection*{LULOS.cs}
\lst{cs}{code/LULOS.cs}

\end{document}