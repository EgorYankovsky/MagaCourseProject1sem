\chapter*{Аннотация}

Отчёт 91 с., 4 ч., 24 рис., 26 табл., 13 источников, 1 прил.

ЧИСЛЕННОЕ МОДЕЛИРОВАНИЕ ЭЛЕКТРОМАГНИТНОГО \\ ПОЛЯ, МЕТОД КОНЕЧНЫХ ЭЛЕМЕНТОВ, МНОГОЭТАПНАЯ СХЕМА РАЗДЕЛЕНИЯ ПОЛЕЙ

Цель работы: разработка программы для численного моделирования нестационарного электромагнитного поля в трёхмерной среде, создаваемого индукционным источником тока, при помощи многоэтапной схемы разделения полей.

В процессе работы был разработан и протестирован  программный модуль численного моделирования электромагнитного поля с помощью многоэтапной схемы разделения полей.

С помощью программы проводилось исследование поведения поля в приповерхностных слоях земной коры с различными значениями удельной электропроводности горизонтально-слоистой среды и аномальных объектов. 