\chapter{Теоретическая часть}

\section{Векторные дифференциальные уравнения второго порядка с разрывными решениями}

Математическая модель, служащая для описания электромагнитного поля в средах с изменяющимся коэффициентом магнитной проницаемости и в ситуациях, когда нельзя пренебрегать влиянием токов смещения, выглядит следующим образом:

\begin{equation} \label{eq_1_1}
	\text{rot} \left( \frac{1}{\mu} \text{rot} \overrightarrow{\textbf{A}} \right) + \sigma \frac{\partial \overrightarrow{\textbf{A}}}{\partial t} + \epsilon \frac{\partial^2 \overrightarrow{\textbf{A}}}{\partial t^2} = \overrightarrow{\textbf{J}}^{\textbf{ст}}.
\end{equation}

Математическая модель электромагнитного поля на основе уравнения (\ref{eq_1_1}) позволяет решать самые сложные задачи электромагнетизма. Она корректно описывает электромагнитные поля в ситуациях, когда среда содержит любые неоднородности с измененными электрическими и магнитными свойствами.

При решении задач с использованием схемы разделения полей, для описания осесимметричной горизонтально-слоистой среды используется следующее уравнение:

\begin{equation*} \label{eq_1_2}
	-\frac{1}{\mu_0} \Delta A_{\varphi} + \frac{A_{\varphi}}{\mu_0 r^2} + \sigma \frac{\partial A_{\varphi}}{\partial t} = J_{\varphi}.
\end{equation*}

В свою очередь, учёт от объектов, имеющих неоднородные значения удельной электропроводности, осуществляется за счёт математической модели, описываемой уравнением (\ref{eq_1_3})

\begin{equation} \label{eq_1_3}
	\text{rot} \left( \frac{1}{\mu_0} \text{rot} \overrightarrow{\textbf{A}}^+ \right) + \sigma \frac{\partial \overrightarrow{\textbf{A}}^+}{\partial t} = \left( \sigma - \sigma_{\text{n}} \right) \overrightarrow{\textbf{E}}_{\text{n}}.
\end{equation}

Для тестирования на правильность решения дифференциального уравнения (\ref{eq_1_3}) будем использовать уравнение, правая часть которого представляется в виде вектор-функции $\overrightarrow{\textbf{F}}$, а также будет иметь место быть слагаемое $\gamma \overrightarrow{\textbf{A}}$ в левой части уравнения:

\begin{equation} \label{eq_1_4}
	\text{rot} \left( \frac{1}{\mu_0} \text{rot} \overrightarrow{\textbf{A}} \right) + \gamma \overrightarrow{\textbf{A}} + \sigma \frac{\partial \overrightarrow{\textbf{A}}}{\partial t} = \overrightarrow{\textbf{F}}.
\end{equation}


\section{Вариационная постановка}

Будем считать, что на границе $S = S_1 \cup S_2$ расчётной области $\Omega$, в которой определено уравнение (\ref{eq_1_4}), заданы краевые условия двух типов:

\begin{equation} \label{eq_1_5}
	\left. \left( \overrightarrow{\textbf{A}} \cross \overrightarrow{\textbf{n}}  \right) \right|_{S_1} = \overrightarrow{\textbf{A}}^g \cross \overrightarrow{\textbf{n}},
\end{equation}

\begin{equation} \label{eq_1_6}
	\left. \left( \frac{1}{\mu} \text{rot} \overrightarrow{\textbf{A}} \cross \overrightarrow{\textbf{n}}  \right) \right|_{S_1} = \overrightarrow{\textbf{H}}^{\Theta} \cross \overrightarrow{\textbf{n}}.
\end{equation}

Тогда эквивалентная вариационная формулировка в форме Галёркина для уравнения (\ref{eq_1_4}) без производной по времени, и с учётом краевых условий (\ref{eq_1_5}) - (\ref{eq_1_6}) имеет вид:

\begin{equation} \label{eq_1_7}
\begin{gathered}
	\int \limits_{\Omega} \frac{1}{\mu_0} \text{rot} \overrightarrow{\textbf{A}} \cdot \text{rot} \overrightarrow{\textbf{$\Psi$}} \, d \Omega + \int \limits_{\Omega} \gamma \overrightarrow{\textbf{A}} \cdot \overrightarrow{\textbf{$\Psi$}} \, d \Omega = \int \limits_{\Omega} \overrightarrow{\textbf{F}} \cdot \overrightarrow{\textbf{$\Psi$}} \, d \Omega + \\ + \int \limits_{S_2} \left( \overrightarrow{\textbf{H}}^{\Theta} \cross \overrightarrow{\textbf{n}} \right) \cdot \overrightarrow{\textbf{$\Psi$}} \, d S \qquad \forall \overrightarrow{\textbf{$\Psi$}} \in H^{rot}_0.
\end{gathered}
\end{equation}


\section{Конечноэлементная дискретизация}

На шестиграннике базисные вектор-функции удобней строить с помощью шаблонного элемента. Обычно в качестве такого берут кубик $\Omega^E \in \left[-1, 1\right] \cross \left[-1, 1\right] \cross \left[-1, 1\right]$ при использовании базиса лагранжева или иерархического типа.

Пусть у нас имеется произвольный шестигранник $\Omega_m$ с вершинами $\left(\hat{x}_i, \hat{y}_i, \hat{z}_i\right), i = 1...8$. Тогда отображение шаблонного кубика $\Omega^E$ в шестигранник $\Omega_m$ будет задаваться соотношениями:

\begin{equation} \label{eq_1_8}
	x = \sum_{i=1}^8 \hat{\varphi}_i (\xi, \eta, \zeta)\hat{x}_i, \qquad  y = \sum_{i=1}^8 \hat{\varphi}_i (\xi, \eta, \zeta)\hat{y}_i, \qquad  z = \sum_{i=1}^8 \hat{\varphi}_i (\xi, \eta, \zeta)\hat{z}_i,
\end{equation}
где $\hat{\varphi}_i (\xi, \eta, \zeta)$ - стандартные скалярные трилинейные базисные функции, определённые на шаблонном элементе $\Omega^E$.

Отображение базисных вектор-функций $\hat{\varphi}_i (\xi, \eta, \zeta)$ шаблонного элемента $\Omega^E$ на шестигранник $\Omega_m$ можно определить следующим образом:

\begin{equation} \label{eq_1_9}
	\hat{\text{$\psi$}}_i (x, y, z) = \textbf{J}^{-1} \hat{\varphi}_i (\xi(x, y, z), \eta(x, y, z), \zeta(x, y, z)),
\end{equation}
где 
\begin{equation} \label{eq_1_10}
	\textbf{J} = 
	\begin{pmatrix}
		\frac{\partial x}{\partial \xi} && \frac{\partial y}{\partial \xi} && \frac{\partial z}{\partial \xi} \\
		
		\frac{\partial x}{\partial \eta} && \frac{\partial y}{\partial \eta} && \frac{\partial z}{\partial \eta} \\
		
		\frac{\partial x}{\partial \zeta} && \frac{\partial y}{\partial \zeta} && \frac{\partial z}{\partial \zeta}
	\end{pmatrix}
\end{equation}
 – функциональная матрица преобразования координат, переводящего кубик $\Omega^E$ в шестигранник $\Omega_m$.

\section{Построение матриц масс, жёсткости и вектора правой части на шестигранниках}

В силу сложной геометрии выпуклых шестигранников, расчёт локальных матриц удобнее проводить на отображении конечного элемента $\Omega_m$ в мастер-элемент $\Omega_E$, представляющий из себя куб размером $\Omega^E = [-1, 1]_x \times [-1, 1]_y \times [-1, 1]_z$. Тогда матрица жёсткости будет рассчитываться по формуле:

\begin{equation} \label{eq_1_11}
\begin{gathered}
	\hat{\text{G}}_{ij} = \int \limits_{\Omega_e} \frac{1}{\mu_0} \text{rot}  \hat{\text{$\psi$}}_i \cdot \text{rot} \hat{\text{$\psi$}}_j d \Omega = \\ = \int \limits_{-1}^1 \int \limits_{-1}^1 \int \limits_{-1}^1 \frac{1}{\mu_0} \frac{1}{|J|} \left( \textbf{J}^{\text{T}} \text{rot} \hat{\varphi}_i \right) \cdot \left( \textbf{J}^{\text{T}} \text{rot} \hat{\varphi}_j \right) \, d \xi d \eta d \zeta
\end{gathered},
\end{equation}
а матрица масс, в свою очередь, по формуле:

\begin{equation} \label{eq_1_12}
	\begin{gathered}
		\hat{\text{M}}_{ij} = \int \limits_{\Omega_e} \gamma  \hat{\text{$\psi$}}_i \cdot \hat{\text{$\psi$}}_j d \Omega = \int \limits_{-1}^1 \int \limits_{-1}^1 \int \limits_{-1}^1 \gamma  \left( \textbf{J}^{\text{-1}} \hat{\varphi}_i \right) \cdot \left( \textbf{J}^{\text{-1}} \hat{\varphi}_j \right) |J| \, d \xi d \eta d \zeta
	\end{gathered}.
\end{equation}

При расчёте локального вектора правой части будем использовать формулу:

\begin{equation} \label{eq_1_13}
	\begin{gathered}
		\hat{\textbf{b}}^{\textbf{J}^{\textbf{CT}}} = \hat{\textbf{M}} \, \hat{\textbf{f}}.
	\end{gathered}
\end{equation}
она универсальна и применима для любой геометрии конечного элемента и любыми базисными функциями любого порядка на нём.

Для произвольных шестигранников интегралы в соотношениях (\ref{eq_1_11}) - (\ref{eq_1_12}) берутся численно. Соответствующая схема вычисления значения интеграла от функции $f(\xi, \eta, \zeta)$ по единичной области $\Omega_E$ выглядит следующим образом:
\begin{equation} \label{eq_1_14}
	\begin{gathered}
		\int \limits_{-1}^1 \int \limits_{-1}^1 \int \limits_{-1}^1 f \left(\xi, \eta, \zeta\right) d\xi d\eta d\zeta \approx \sum_{k=1}^{n} \sum_{l=1}^{n} \sum_{r=1}^{n} f\left(t_k, t_l, t_r\right).
	\end{gathered}
\end{equation}

\section{Учёт краевых условий}

При решении уравнения (\ref{eq_1_3}) с использованием векторного МКЭ базисные вектор-функции $\overrightarrow{\psi}_i$ строятся так, что все базисные функции конечномерного пространства $\textbf{V}^\text{rot}$ с индексами $i \in N_0$ имеют нулевые касательные к $S_1$ составляющие (они образуют базис подпространства $\textbf{V}^\text{rot}_0$). Поэтому для выполнения однородных главных краевых условий достаточно к $n_0$ уравнениям системы (\ref{eq_1_7}) добавить $n - n_0$ уравнений вида:

\begin{equation*} \label{eq_1_15}
	\begin{gathered}
		q_i = 0, \hspace{1cm} i \in N \backslash N_0.
	\end{gathered}
\end{equation*}

Для учёта неоднородного краевого условия (\ref{eq_1_5}), как и в случае однородного, к $n_0$ уравнениям добавляется ещё $n - n_0$ линейных (относительно весов $q_i$) уравнений, которые и должны обеспечить необходимую близость на $S_1$ касательных составляющих приближённого решения $\overrightarrow{\textbf{A}}^h = \sum \limits_{j \in N} q_j^n \, \overrightarrow{\psi}_j$ к касательным составляющим вектора $\overrightarrow{\textbf{A}}^g$.

Однако для учёта неоднородных краевых условий в векторном МКЭ можно использовать гораздо более простую и удобную для реализации процедуру. Основная её идея заключается в том, что вектор-функция $\overrightarrow{\textbf{A}}^g$ из правой части краевых условий (\ref{eq_1_5}) заменяется некоторым интерполянтом $\overrightarrow{\textbf{A}}^{g,h}$, представленными в виде линейной комбинации базисных вектор-функций: 

\begin{equation} \label{eq_1_16}
	\begin{gathered}
		\overrightarrow{\textbf{A}}^{g,h} = \sum \limits_j q_j^g \, \overrightarrow{\psi_j}.
	\end{gathered}
\end{equation}

Веса $q_j^g$ в разложении (\ref{eq_1_16}) можно найти следующим образом. Поскольку в векторном МКЭ базисные функции $\overrightarrow{\psi}_i$ строятся так, что на поверхности $S_1$ ненулевые касательные имеют только $n-n_0$ базисных вектор-функций с номерами $i \in N \backslash N_0$, при этом для каждой из таких вектор-функций на поверхность $S_1$ существует точка $\left(x_i, y_i, z_i\right)$ и проходящий через эту точку касательный к поверхности $S_1$ вектор $\overrightarrow{\tau}_i$ такой, что

\begin{equation} \label{eq_1_17}
	\begin{gathered}
		\overrightarrow{\psi}_i \left(x_i, y_i, z_i\right) \cdot \overrightarrow{\tau}_i \neq 0, \hspace{1cm} \overrightarrow{\psi}_j \left(x_i, y_i, z_i\right) \cdot \overrightarrow{\tau}_i = 0, \hspace{1cm} \forall j \neq i.
	\end{gathered}
\end{equation}

Домножим левую и правую части уравнения (\ref{eq_1_16}) скалярно на вектор $\overrightarrow{\tau}_i$ в точке $\left(x_i, y_i, z_i\right)$ с учётном (\ref{eq_1_17}) получим 

\begin{equation*} \label{eq_1_18}
	\begin{gathered}
		\overrightarrow{\textbf{A}}^{g,h} \left(x_i, y_i, z_i\right) \cdot \overrightarrow{\tau}_i = q_i^g \, \overrightarrow{\psi}_i \left(x_i, y_i, z_i\right) \cdot \overrightarrow{\tau}_i.
	\end{gathered}
\end{equation*}

Полагая, что в точках $\left(x_i, y_i, z_i\right)$ значения проекции на $\overrightarrow{\tau}_i$ интерполянта $\overrightarrow{\textbf{A}}^{g,h}$ должны совпадать со значениями проекций на $\overrightarrow{\tau}_i$ вектор-функции $\overrightarrow{\textbf{A}}^g$, а также потребовав равенства касательных составляющих $\overrightarrow{\textbf{A}}^h \cross \overrightarrow{\textbf{n}}$ искомой вектор-функции $\overrightarrow{\textbf{A}}^h$ к касательным составляющим $\overrightarrow{\textbf{A}}^{g,h} \cross \overrightarrow{\textbf{n}}$ вектор-функции

\begin{equation*} \label{eq_1_19}
	\begin{gathered}
		\overrightarrow{\textbf{A}}^{g,h} = \sum \limits_{i \in N \backslash N_0} q_j^g \, \overrightarrow{\psi_j},
	\end{gathered}
\end{equation*}
получим следующее выражение для учёта неоднородного главного краевого условия:

\begin{equation*} \label{eq_1_20}
	\begin{gathered}
		q_i = \frac{\overrightarrow{\textbf{A}}^g \left(x_i, y_i, z_i\right) \cdot \overrightarrow{\tau}_i}{\overrightarrow{\psi}_i \left(x_i, y_i, z_i\right) \cdot \overrightarrow{\tau}_i}, \hspace{1cm} i \in N \backslash N_0.
	\end{gathered}
\end{equation*}

\section{Решатель СЛАУ PARDISO}

PARDISO (Parallel Direct Sparse Solver) — это высокопроизводительная библиотека для решения систем линейных алгебраических уравнений (СЛАУ) с разреженными матрицами. Хотя PARDISO изначально разработан как прямой решатель (direct solver), он также поддерживает итерационные методы (iterative methods) для решения задач, где прямое решение может быть неэффективным или требовать слишком много ресурсов.

К основным особенностям PARDISO можно отнести несколько пунктов. Во-первых решатель оптимизирован для работы с разреженными матрицами, которые часто возникают в задачах математического моделирования, физики, инженерии и других областях. Разреженные матрицы содержат большое количество нулевых элементов, и PARDISO эффективно использует эту структуру для уменьшения вычислительных затрат.

Во-вторых это гибкость в выборе методов. PARDISO поддерживает как прямые методы (например, LU- или Cholesky-разложение), так и итерационные методы (например, метод сопряжённых градиентов, GMRES). Это позволяет выбирать наиболее подходящий метод в зависимости от задачи.

В-третьих имеется поддержка параллельных вычислений. PARDISO разработан для эффективного использования многопроцессорных систем и графических ускорителей (GPU). Это делает его одним из самых быстрых решателей для задач с большими разреженными матрицами.

Также при использовании решателя имеется поддержка различных типов матриц. PARDISO работает с симметричными, несимметричными, положительно определёнными и неопределёнными матрицами. Это делает его универсальным инструментом для широкого круга задач.

И наконец имеется возможность интеграции с популярными платформами. PARDISO легко интегрируется с такими языками программирования, как C, C++, Fortran, а также с математическими пакетами, например, MATLAB.

Хотя PARDISO изначально ориентирован на прямые методы, он также поддерживает итерационные методы для решения СЛАУ. Это особенно полезно для задач с очень большими разреженными матрицами, где прямое решение может быть слишком затратным по памяти или времени. Например, метод сопряжённых градиентов, который эффективен для симметричных положительно определённых матриц и если матрица хорошо обусловлена, то сходится за небольшое количество итераций. Также можно использовать метод обобщённый минимальных невязок (GMRES). Он подходит для несимметричных матриц. Использует процесс Арнольди для построения ортогонального базиса. Метод бисопряжённых градиентов подходит для несимметричных матриц, более устойчив, чем классический метод BiCG. Также имеется метод минимальных невязок (MINRES) в котором используется для симметричных неопределённых матриц.

К преимуществам использования PARDISO можно отнести высокую производительность (PARDISO использует современные алгоритмы и оптимизации для достижения максимальной скорости решения), экономию памяти (благодаря работе с разреженными матрицами PARDISO минимизирует использование памяти), гибкость (поддержка как прямых, так и итерационных методов позволяет адаптировать решение под конкретную задачу) и масштабируемость (PARDISO эффективно использует многопроцессорные системы, что делает его пригодным для решения задач высокой размерности).

% Далее глава пркатической части.