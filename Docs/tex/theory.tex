\chapter{Теоретическая часть}

\section{Векторные дифференциальные уравнения второго порядка с разрывными решениями}

Математическая модель, служащая для описания электромагнитного поля в средах с изменяющимся коэффициентом магнитной проницаемости и в ситуациях, когда нельзя пренебрегать влиянием токов смещения, выглядит следующим образом (\ref{eq_1_1}):

\begin{equation} \label{eq_1_1}
	\text{rot} \left( \frac{1}{\mu} \text{rot} \overrightarrow{\textbf{A}} \right) + \sigma \frac{\partial \overrightarrow{\textbf{A}}}{\partial t} + \epsilon \frac{\partial^2 \overrightarrow{\textbf{A}}}{\partial t^2} = \overrightarrow{\textbf{J}}^{\textbf{ст}}.
\end{equation}

Математическая модель электромагнитного поля на основе уравнения (\ref{eq_1_1}) позволяет решать самые сложные задачи электромагнетизма. Она корректно описывает электромагнитные поля в ситуациях, когда среда содержит любые неоднородности с измененными электрическими и магнитными свойствами.

При решении задач с использованием схемы разделения полей, для описания осесимметричной горизонтально-слоистой среды используется следующее уравнение (\ref{eq_1_2}):

\begin{equation} \label{eq_1_2}
	-\frac{1}{\mu_0} \Delta A_{\varphi} + \frac{A_{\varphi}}{\mu_0 r^2} + \sigma \frac{\partial A_{\varphi}}{\partial t} = J_{\varphi}.
\end{equation}

В свою очередь, учёт от объектов, имеющих неоднородные значения удельной электропроводности, осуществляется за счёт математической модели, описываемой уравнением (\ref{eq_1_3})

\begin{equation} \label{eq_1_3}
	\text{rot} \left( \frac{1}{\mu_0} \text{rot} \overrightarrow{\textbf{A}}^+ \right) + \sigma \frac{\partial \overrightarrow{\textbf{A}}^+}{\partial t} = \left( \sigma - \sigma_{\text{n}} \right) \overrightarrow{\textbf{E}}_{\text{n}}.
\end{equation}

Для тестирования на правильность решения дифференциального уравнения (\ref{eq_1_3}) будем использовать уравнение (\ref{eq_1_4}), правая часть которого представляется в виде вектор-функции $\overrightarrow{\textbf{F}}$, а также будет иметь место быть слагаемое $\gamma \overrightarrow{\textbf{A}}$ в левой части уравнения:

\begin{equation} \label{eq_1_4}
	\text{rot} \left( \frac{1}{\mu_0} \text{rot} \overrightarrow{\textbf{A}} \right) + \gamma \overrightarrow{\textbf{A}} + \sigma \frac{\partial \overrightarrow{\textbf{A}}}{\partial t} = \overrightarrow{\textbf{F}}.
\end{equation}


\section{Вариационная постановка}

Будем считать, что на границе $S = S_1 \cup S_2$ расчётной области $\Omega$, в которой определено уравнение (\ref{eq_1_4}), заданы краевые условия двух типов:

\begin{equation} \label{eq_1_5}
	\left. \left( \overrightarrow{\textbf{A}} \cross \overrightarrow{\textbf{n}}  \right) \right|_{S_1} = \overrightarrow{\textbf{A}}^g \cross \overrightarrow{\textbf{n}},
\end{equation}

\begin{equation} \label{eq_1_6}
	\left. \left( \frac{1}{\mu} \text{rot} \overrightarrow{\textbf{A}} \cross \overrightarrow{\textbf{n}}  \right) \right|_{S_1} = \overrightarrow{\textbf{H}}^{\Theta} \cross \overrightarrow{\textbf{n}}.
\end{equation}

Тогда эквивалентная вариационная формулировка в форме Галёркина для уравнения (\ref{eq_1_4}) без производной по времени, и с учётом краевых условий (\ref{eq_1_5}) - (\ref{eq_1_6}) имеет вид:

\begin{equation} \label{eq_1_7}
\begin{gathered}
	\int \limits_{\Omega} \frac{1}{\mu_0} \text{rot} \overrightarrow{\textbf{A}} \cdot \text{rot} \overrightarrow{\textbf{$\Psi$}} \, d \Omega + \int \limits_{\Omega} \gamma \overrightarrow{\textbf{A}} \cdot \overrightarrow{\textbf{$\Psi$}} \, d \Omega = \int \limits_{\Omega} \overrightarrow{\textbf{F}} \cdot \overrightarrow{\textbf{$\Psi$}} \, d \Omega + \\ + \int \limits_{S_2} \left( \overrightarrow{\textbf{H}}^{\Theta} \cross \overrightarrow{\textbf{n}} \right) \cdot \overrightarrow{\textbf{$\Psi$}} \, d S \qquad \forall \overrightarrow{\textbf{$\Psi$}} \in H^{rot}_0.
\end{gathered}
\end{equation}


\section{Конечноэлементная дискретизация}

На шестиграннике базисные вектор-функции удобней строить с помощью шаблонного элемента. Обычно в качестве такого берут кубик $\left[-1, 1\right] \cross \left[-1, 1\right] \cross \left[-1, 1\right]$ при использовании базиса лагранжева или иерархического типа.

Пусть у нас имеется произвольный шестигранник $\Omega_m$ с вершинами $\left(\hat{x}_i, \hat{y}_i, \hat{z}_i\right), i = 1...8$. Тогда отображение шаблонного кубика $\Omega^E$ в шестигранник $\Omega_m$ будет задаваться соотношениями:

\begin{equation} \label{eq_1_8}
	x = \sum_{i=1}^8 \hat{\varphi}_i (\xi, \eta, \zeta)\hat{x}_i, \qquad  y = \sum_{i=1}^8 \hat{\varphi}_i (\xi, \eta, \zeta)\hat{y}_i, \qquad  z = \sum_{i=1}^8 \hat{\varphi}_i (\xi, \eta, \zeta)\hat{z}_i,
\end{equation}
где $\hat{\varphi}_i (\xi, \eta, \zeta)$ - стандартные скалярные трилинейные базисные функции, определённые на шаблонном элементе $\Omega^E$.

Отображение базисных вектор-функций $\hat{\varphi}_i (\xi, \eta, \zeta)$ шаблонного элемента $\Omega^E$ на шестигранник $\Omega_m$ можно определить следующим образом:

\begin{equation} \label{eq_1_9}
	\hat{\text{$\psi$}}_i (x, y, z) = \textbf{J}^{-1} \hat{\varphi}_i (\xi(x, y, z), \eta(x, y, z), \zeta(x, y, z)),
\end{equation}
где 
\begin{equation} \label{eq_1_10}
	\textbf{J} = 
	\begin{pmatrix}
		\frac{\partial x}{\partial \xi} && \frac{\partial y}{\partial \xi} && \frac{\partial z}{\partial \xi} \\
		
		\frac{\partial x}{\partial \eta} && \frac{\partial y}{\partial \eta} && \frac{\partial z}{\partial \eta} \\
		
		\frac{\partial x}{\partial \zeta} && \frac{\partial y}{\partial \zeta} && \frac{\partial z}{\partial \zeta}
	\end{pmatrix}
\end{equation}
 - функциональная матрица преобразования координат, переводящего кубик $\Omega^E$ в шестигранник $\Omega_m$.

\section{Построение матриц масс, жёсткости и вектора правой части на шестигранниках}

В силу сложной геометрии выпуклых шестигранников, расчёт локальных матриц удобнее проводить на отображении конечного элемента $\Omega_m$ в мастер-элемент $\Omega_E$, представляющий из себя куб размером $\Omega_E = [-1, 1]_x \times [-1, 1]_y \times [-1, 1]_z$. Тогда матрица жёсткости будет рассчитываться по формуле (\ref{eq_1_11}):

\begin{equation} \label{eq_1_11}
\begin{gathered}
	\hat{\text{G}}_{ij} = \int \limits_{\Omega_e} \frac{1}{\mu_0} \text{rot}  \hat{\text{$\psi$}}_i \cdot \text{rot} \hat{\text{$\psi$}}_j d \Omega = \\ = \int \limits_{-1}^1 \int \limits_{-1}^1 \int \limits_{-1}^1 \frac{1}{\mu_0} \frac{1}{|J|} \left( \textbf{J}^{\text{T}} \text{rot} \hat{\varphi}_i \right) \cdot \left( \textbf{J}^{\text{T}} \text{rot} \hat{\varphi}_j \right) \, d \xi d \eta d \zeta
\end{gathered}
\end{equation}
матрица масс, в свою очередь, по формуле (\ref{eq_1_12}):

\begin{equation} \label{eq_1_12}
	\begin{gathered}
		\hat{\text{M}}_{ij} = \int \limits_{\Omega_e} \gamma  \hat{\text{$\psi$}}_i \cdot \hat{\text{$\psi$}}_j d \Omega = \int \limits_{-1}^1 \int \limits_{-1}^1 \int \limits_{-1}^1 \gamma  \left( \textbf{J}^{\text{-1}} \hat{\varphi}_i \right) \cdot \left( \textbf{J}^{\text{-1}} \hat{\varphi}_j \right) |J| \, d \xi d \eta d \zeta
	\end{gathered}
\end{equation}

При расчёте локального вектора правой части будем использовать формулу:

\begin{equation} \label{eq_1_13}
	\begin{gathered}
		\hat{\textbf{b}}^{\textbf{J}^{\textbf{CT}}} = \hat{\textbf{M}} \, \hat{\textbf{f}}.
	\end{gathered}
\end{equation}
она универсальна и применима для любой геометрии конечного элемента и любыми базисными функциями любого порядка на нём.
% Далее глава пркатической части.