\chapter*{Введение}

\addcontentsline{toc}{chapter}{Введение}

С увеличением потребности в природных ресурсах с конца $\text{XIX}$-ого -- начала $\text{XX}$-ого века развивались методы поиска и исследования земных пород и руд. Одним из наиболее распространенных методов является электроразведка. Конкретно в электроразведке сейчас насчитывается свыше пятидесяти различных методов и модификаций, предназначенных как для глубинных исследований, так и для изучения верхней части разреза земной коры. Одним из наиболее распространённым в наши дня является индукционный метод. Его принцип заключается в следующем: под влиянием переменного электрического или магнитного поля в земле за счет феномена магнитной индукции возникает электромагнитное поле. Зная точно параметры источника поля, можно измерять различные электрические и магнитные компоненты индуцированного поля, восстанавливая по ним значения параметров среды.

Помимо возможности нахождения параметров среды, что является обратной задачей по определению, можно изучать и поведение самого индукционного поля. Зная значения силы тока на источнике и физические характеристики верхних слоёв земной коры можно смоделировать электромагнитное поле и изучать характер его поведения в зависимости, например, от количества неоднородностей в земной среде или параметров горизонтально-слоистой среды Земли \cite{1}.

Однако на тот момент аналитических методов для моделирования электромагнитных полей в трёхмерном пространстве не существовало. Достаточно быстро решать такие задачи стало возможно лишь с развитием ЭВМ -- в 30-х -- 70-х годах $\text{XX}$--го века. Использование компьютеров позволило численно моделировать сложные поля, для которых было необходимо преобразовывать их таким образом, чтобы разделить аномалии в зависимости от глубины расположения источников поля и обособлять такие изменения поля, которые соответствуют аномалиям тел простейших форм \cite{2}. 

В это же время для решения сложных уравнений в частных производных, описывающих большинство всех физический процессов в природе, были разработаны такие методы как: метод конечных элементов (МКЭ), метод конечных объемов (МКО) и метод конечных разностей (МКР). МКР известен своей простотой в реализации и используется при решении очень большого класса задач математической физики. Однако данный метод имеет ряд существенных недостатков, например, невозможность решения задачи на объектах, имеющих сложную геометрию. Куда более универсальными численными методами решения задач математической физики являются МКО и МКЭ. Оба метода применяются для решения задач аэрогидродинамики, электростатики, упругости и прочности материалов. Данные методы позволяют решать задачи для предметов любой геометрической сложности, а также находить значение искомой функции в любом месте расчетной области без дополнительного применения интерполяции полученного результата. Сегодня есть очень много профессионального коммерческого программного обеспечения для анализа проблем методом конечных элементов или контрольных объёмов, например американская ANSYS или российская Штуцер-МКЭ.

При численном моделировании электромагнитных полей, большее предпочтение отдаётся МКЭ нежели МКО, поскольку при решении задачи методом конечных элементов можно воспользоваться модификацией данного метода - векторным МКЭ. Данная модификация позволяет куда лучше найти решение задачи при наличии, например, разрывности напряженности электрического поля и удельной электрической проводимости среды.

В данной работе будет рассмотренна возможность расчёта нестационарного электромагнитного поля, создаваемого индукционным источником при использовании как скалярного, так и векторного МКЭ, а также с применением многоэтапной технологии разделения полей.

Исследования будем проводить на области размером $\Omega \in [-55000; 55000]_x \times [-55000; 55000]_y \times [-25000; 25000]_z$. Данная область включает в себя несколько разных слоев земных пород, имеющих различные удельные значения физических величин. Также рассмотрим процесс добавления нескольких аномальных пород, характерных различным земным рудам.

При написании программы для выпускной квалификационной работы использовались следующие языки программирования: для математических расчётов -- C$\#$ 12 на платформе .NET 8.0, для визуализации полученных результатов -- Python 3.12.2 с пакетами matplotlib версии 3.8.2 и numpy версии 1.26.4. 
